% !TeX encoding = UTF-8
% !BIB TS-program = biber
% !TeX TS-program = xelatex
% This is file `main.tex'.
% Copyright (C) 2021-2024 by Linrong Wu.
% Version: 2024/04/16 v1.3c (Original Version: 2021/11/30 v1.0a).
% 本文件为 SCU_Beamer_Slide-demo 主文件源文件.
% !使用前请阅读用户手册.

% ================ %
%      导言区      %
% ================ %
\documentclass[hyperref, UTF8, CJK]{beamer}
%\special{dvipdfmx:config z 0}

% \colorlet{PrimaryC}{white}
% \colorlet{BackgroundC}{black}
% \colorlet{NomalTextC}{white}

% --------SCU Beamer 模板宏包--------
% ----------------
\usetheme[ColorDisplay=Custom, ContentMuticols=ture,
	Background=false]{scu}
	

% --------不要调用的宏包--------
% ----------------
%\usepackage{pstricks} % pstricks: 绘图. (请勿调用, 本模板中调用的 adjustbox 宏包与之冲突)
%\usepackage{geometry} % geometry: 页面设置. (请勿在 Beamer 中调用)

% --------宏包调用--------
% ----------------
\usepackage{transparent}
\usepackage[autoplay]{animate}
\usepackage{tikz}
\usetikzlibrary{shadings,arrows,calc,decorations.pathmorphing,patterns}
\usepackage{array}
\usepackage{algorithm,algorithmic}
\usepackage{amsmath,amsfonts,amssymb} % math equations, symbols
\usepackage{mathtools}
\usepackage[english]{babel}
\usepackage{color}      % color content
\usepackage{url}        % hyperlinks
\usepackage{multicol,multirow}
\usepackage{ulem} % ulem: 添加线.
\usepackage{booktabs}
\usepackage{cprotect}
\usepackage{makecell}
\usepackage{listings}
\usepackage{subcaption}
\usepackage{varioref,cleveref}
%\usepackage[active,tightpage]{preview}
%\PreviewEnvironment{pspicture}

% --------newcommand 区--------
% 建议在此定义常用命令.
% ----------------
\newcommand{\fverb}[1]{\texttt{#1}}

% --------封面信息输入--------
% [<in footline>], {<in title page>} 方括号内容显示在页脚, 花括号内容为全称显示在封面.
% ----------------
\title[四川大学虚拟偶像研究 | Beamer模板使用答辩]{四川大学虚拟偶像研究}
\subtitle{Beamer模板使用答辩} % subtitle 未设置页脚显示项, 请在 title 中设置.
\author[我不卷, 你才卷]{马老卷\inst{1} \and 马小卷\inst{2}}
\institute{%
	\inst{1} 混元形翼太极门
	~(\textit{MaLJFake@taichi.hunyuan})
	\vspace*{-6pt} \and
	\inst{2} Management Science, Business School, Sichuan University
	\\(\textit{MaXJFake@scu.edu.cn})
}
\date{April 16th, 2024}

% ---------------- %
%      正文区      %
% ---------------- %
\begin{document}

% --------总目录--------
% 可注释.
% ----------------
%	\begin{frame}{目录}
%		%\transfade%淡入淡出 
%		\tableofcontents % 显示目录.
%	\end{frame}

% --------节: 声明--------
% ----------------
\section{引言}
\subsection{研究现状}
\begin{frame}{关于本模板}%=0.85
	\begin{itemize}
		\item 创建初衷:
		\begin{itemize}
			\item 编者本人对\LaTeX{} 稍有涉足, 这也是编者的首个Beamer模板, 模板创建源于本学院李璐老师提出的PPT修改意见;
			\item 项目也源于制作者本人的兴趣, 但本人对\LaTeX{} 的了解仍处在较浅层次, 故编写的模板可能会存在不兼容、编译后版式错位等现象;
		\end{itemize}
		\item 项目地址:
		\begin{itemize}
			\item 使用前请前往下列地址中查看模板版本!
			\item \faGithub\enspace{\color{scublue}\url{https://github.com/FvNCCR228/SCU_Beamer_Slide-demo}}
			\item Gitee: \color{scublue}\url{https://gitee.com/NCCR/SCU_Beamer_Slide-demo}
		\end{itemize}
		%			\framebreak
		%		\pagebreak
		\item 联系方式:
		\begin{itemize}
			\item 制作者: linrong.wu.interact@outlook.com
			
			OR linrong.wu.work@outlook.com
		\end{itemize}
	\end{itemize}
	目前仅在制作者的Windows 11系统上编译通过, Overleaf编译失败(不是完全失败, 我很伤心);
\end{frame}

\begin{frame}{使用注意}{雪豹闭嘴}
	\begin{itemize}
		\item<1-> \LaTeX 编辑器:
		\begin{itemize}
			\item<1-> 本地: TeX Live (推荐\href{https://mirrors.tuna.tsinghua.edu.cn/CTAN/systems/texlive/Images/}{\color{scublue}清华大学开源软件镜像站}安装最新版)配合TeXstudio或VS Code使用. TeX Live安装时间极长, 请各位做好心理准备. 此外Apple设备IDE平台建议知乎;
			\item<1-> 在线: Overleaf平台, TeXPage平台.
		\end{itemize}
		\item<2-> \LaTeX 相关插件:
		\begin{itemize}
			\item<2-> 表格转换: Excel2\LaTeX~(\href{https://www.ctan.org/tex-archive/support/excel2latex/}{\color{scublue}CTAN Excel2\LaTeX});
			\item<2-> 在线公式: \href{https://www.latexlive.com/}{\color{scublue}LaTeX公式编辑器}, \href{https://mathpix.com/}{\color{scublue}
				Mathpix}~\&~\href{https://mathf.itewqq.cn/}{\color{scublue}图片在线转LaTeX}.
		\end{itemize}
		\item<3-> \color{scured}!! 编译相关:
		\begin{itemize}
			\item<3-> \alert{!! 请使用UTF-8格式, 设置XeLaTeX和Biber进行编译};
			\item<3-> 在线编辑请上传整个工作文件夹, 否则会出现严重错误(Bug遍地飞);
			\item<3-> \color{scured}!! 对\LaTeX 不熟悉的情况下, 请勿轻易改动".sty"文件(宏包文件)中代码, 也可按照文件中注释进行实验性修改(注意保留备份).
		\end{itemize}
		\item<4-> \color{scured} 建议使用Adobe Acrobat作为PDF浏览器(Ctrl+L全屏食用效果良好).
	\end{itemize}
\end{frame}

% --------节: 研究分析--------
% ----------------
\section{研究分析}
\subsection{字}
\begin{frame}[fragile]{添加线}
	\begin{multicols}{2}
		\verb|\uline|\hfill 下划线\qquad\uline{混}\\
		\verb|\uuline|\hfill 双下划线\qquad\uuline{元}\\
		\verb|\uwave|\hfill 波浪线\qquad\uwave{形}\\
		\verb|\sout|\hfill 删除线\qquad\sout{翼}\\
		\verb|\xout|\hfill 斜删除线\qquad\xout{太}\\
		\verb|\dashuline|\hfill 虚线\qquad\dashuline{极}\\
		\verb|\dotuline|\hfill 加点\qquad\dotuline{门}
	\end{multicols}
\end{frame}

\subsection{图, 表, 代码}
\begin{frame}{图}
	\begin{figure}[h]
		\begin{subfigure}{.4\columnwidth}
			\centering
			\includegraphics%
			[width=.3\columnwidth]{stop-rd.pdf}
			\caption{白天的暂停}
		\end{subfigure}
		\quad
		\begin{subfigure}{.4\columnwidth}
			\centering
			\includegraphics%
			[width=.3\columnwidth]{stop-gn.pdf}
			\caption{晚上的暂停}
		\end{subfigure}
		\caption{掌门常用的暂停}
	\end{figure}
	\begin{figure}[h]
		\begin{minipage}[t]{.4\columnwidth}
			\centering
			\includegraphics%
			[width=.3\columnwidth]{stop-rd.pdf}
			\caption{掌门白天的暂停}
			\label{fig:ZhangmenBtdzt}
		\end{minipage}
		\quad
		\begin{minipage}[t]{.4\columnwidth}
			\centering
			\includegraphics%
			[width=.3\columnwidth]{stop-gn.pdf}
			\caption{掌门晚上的暂停}
			\label{fig:ZhangmenWsdzt}
		\end{minipage}
	\end{figure}
\end{frame}

\cprotEnv\begin{frame}
	\frametitle{表}
	表格太麻烦了, 掌门说摸摸鱼, 编者觉得不错, 丢一个三线表示例. 当然也可以看看这个手册前面部分表格的源码.
	\begin{table}[htbp]
		\centering
		\caption{一些国风音乐}
		\label{tab:YixieGfyy}
		\begin{tabular}{rlc}
			\toprule
			作曲家 & 歌名 & 门中喜欢的友人 \\
			\midrule
			李志辉 & 小桥流水人家 & 门主 \\
			林海 & 无羁(器乐版) & 初号 \\
			吕秀龄 & 逆伦 & 小初 \\
			麦振鸿 & 从来只有一个人 & 编者(假的) \\
			\bottomrule
		\end{tabular}
	\end{table}
\end{frame}

\subsection{代码环境}
\begin{frame}[fragile]{代码环境演示}
	\onslide<2>
	\begin{scucode}{A welcome program.}[cpphelloworld]{c}
#include <iostream>
int main()
{
	std::cout << "Hello World!" << std::endl;
	std::cin.get();
}
	\end{scucode}
	\onslide<1>
	\begin{scucode}{A welcome program.}[chelloworld]{c}
#include <stidio.h>
int main()
{
	printf("Hello World!");
	return 0;
}
	\end{scucode}
\end{frame}

\subsection{数学}
\begin{frame}[fragile,allowframebreaks]{数学环境}
	\begin{scutheorem}{切比雪夫大数率}[QiebiXfdsl]
		对独立随机变量序列$\{X_k\}$, 若$E(X_k)$, $D(X_k)$都存在, $k=1,2,\cdots$, 且有常数$C$, 使得$D(X_k)\leq C$, $k=1,2,\cdots$, 则有
		\begin{equation}
		\dfrac{1}{n} \sum_{k=1}^{n} X_k - \dfrac{1}{n} \sum_{k=1}^{n} E(X_k) \stackrel{\;P\;}{\longrightarrow} 0
		\end{equation}
	\end{scutheorem}
	\begin{scuproof}{}
		请读者自证.
	\end{scuproof}
	\begin{scuexample}{形翼门的规模}[HunyuanXytjmdgm]
		本门昨天去了80个人打水, 今天去了79个人打水, 本门的规模有多大?
	\end{scuexample}
	\begin{scualgorithm}{怎么写Beamer模板}[ZengmoXBeamerzs]
		\begin{algorithmic}[1]
			\REQUIRE 一点点\LaTeX 知识, 不要太信任百度
			\ENSURE 不知道怎么搞
			\STATE 问门主, 肯定不知道
			\STATE 问初号, 当然不知道
			\STATE 问小初, 还是不知道
			\RETURN 算了, 不问了, 都是不知道
		\end{algorithmic}
	\end{scualgorithm}
	\begin{scudefinition}{马老卷}[MalaoJ]
		是形翼门的打砸工, 直系上峰是马凡王, 入门改姓马, 自称老卷, 实则不卷.
	\end{scudefinition}
	\begin{scuaxiom}{皮亚诺公理}[PiyaNgl]
		略.
	\end{scuaxiom}
	\begin{scuproperty}{刚体的性质}[GangtiDxz]
		刚体是个理想模型. 虽然理想但是还是那么难整, 进动和章动就不会了.
	\end{scuproperty}
	\begin{scuproposition}{不确定性原理}[BuqueDxyl]
		粒子的位置与动量不可同时被确定, 位置的不确定性与动量的不确定性遵守不等式
		\begin{equation}
		\Delta x \Delta p \geq \dfrac{h}{4\pi}
		\end{equation}
		其中$h$为普朗克常数.
	\end{scuproposition}
	\begin{sculemma}{卷王森林法则}[JuanwangSlfz]
		源自未知高校学生, 此处略.
	\end{sculemma}
	\begin{scucorollary}{狼人杀的重要性}[LangrenSdzyx]
		编者实习时听公司导师说面试有可能是趣味性游戏, 狼人杀感觉很符合, 所以玩狼人杀吧.
	\end{scucorollary}
	\begin{scuremark}{}[Zhu]
		\vref{coro:LangrenSdzyx}, 只是推论, 编者瞎说的.
	\end{scuremark}
	\begin{scucondition}{面试狼人杀的条件}[MianshiLrsdtj]
		\vref{coro:LangrenSdzyx}, 此推论有条件, 即真有公司面试用狼人杀.
	\end{scucondition}
	\begin{scuconclusion}{爱废话的编者}[AifeiHdbz]
		由上述可知: 编者爱废话.
	\end{scuconclusion}
	\begin{scuassumption}{编者不会废话}[BianzheBhfh]
		我们可以假设编者不会废话, 假设成立, 编者当然不会废话.
	\end{scuassumption}
\end{frame}

\begin{frame}[fragile,allowframebreaks]{数学公式}
	麦克斯韦分布函数$f(v) = \dfrac{\mathrm{d}N}{N\,\mathrm{d}v} = 4\pi \Big(\dfrac{\mu}{2\pi kT}\Big)^{3/2} v^2 \mathrm{exp}\Big(-\dfrac{\mu v^2}{2kT}\Big)$.\\[1ex]
	最概然速率\[v_p = \sqrt{\dfrac{2kT}{\mu}} = \sqrt{\dfrac{2RT}{M}}\]其中$R$是气体常数, $M = N_A \mu$是物质的摩尔质量.\\[1ex]
		\begin{equation*}
			\bar{v} = \int_0^\infty vf(v)\,\mathrm{d}v = \sqrt{\dfrac{8kT}{\pi\mu}} = \sqrt{\dfrac{8RT}{\pi M}}
		\end{equation*}
	方均根速率
		\begin{equation}
		v_{rms} = \Big(\int_0^\infty v^2f(v)\,\mathrm{d}v\Big)^{1/2} = \sqrt{\dfrac{3kT}{\mu}} = \sqrt{\dfrac{3RT}{M}}
		\end{equation}
	$\symbb{R}\symbbit{R}\symcal{R}\symscr{R}\symfrak{R}\symsfup{R}\symsfit{R}\symbfsf{R}\symbfup{R}\symbfit{R}\symbfcal{R}\symbfscr{R}\symbffrak{R}\symbfsfup{R}\symbfsfit{R}$\\[1ex]
	多行公式 
		\begin{multline}
			A=\lim_{n\rightarrow\infty}\Delta x\left(a^{2}+\left(a^{2}+2a\Delta x+\left(\Delta x\right)^{2}\right)\right.\label{eq:reset}\\
			+\left(a^{2}+2\cdot2a\Delta x+2^{2}\left(\Delta x\right)^{2}\right)\\
			+\left(a^{2}+2\cdot3a\Delta x+3^{2}\left(\Delta x\right)^{2}\right)\\
			+\ldots\\
			\left.+\left(a^{2}+2\cdot(n-1)a\Delta x+(n-1)^{2}\left(\Delta x\right)^{2}\right)\right)\\
			=\frac{1}{3}\left(b^{3}-a^{3}\right)
		\end{multline}\\[1ex]
	质能方程
		\begin{align}
			E=m&c^2 & &E=mc^2\\
			E=~&mc^2 & E&=mc^2 \notag \\
			E&=mc^2 & E=~&mc^2\\
			&E=mc^2 & E=m&c^2
		\end{align}
	?
		\begin{equation}
			\begin{aligned}
			\dfrac{\mathrm{d}}{\mathrm{d}t} \symbfit{f} &= \dfrac{\mathrm{d}f_x}{\mathrm{d}t} \symbfit{\hat{i}} + \dfrac{\mathrm{d}\symbfit{\hat{i}}}{\mathrm{d}t} f_x + \dfrac{\mathrm{d}f_y}{\mathrm{d}t} \symbfit{\hat{j}} + \dfrac{\mathrm{d}\symbfit{\hat{j}}}{\mathrm{d}t} f_y + \dfrac{\mathrm{d}f_z}{\mathrm{d}t} \symbfit{\hat{k}} + \dfrac{\mathrm{d}\symbfit{\hat{k}}}{\mathrm{d}t} f_z \\
			&= \dfrac{\mathrm{d}f_x}{\mathrm{d}t} \symbfit{\hat{i}} + \dfrac{\mathrm{d}f_y}{\mathrm{d}t} \symbfit{\hat{j}} + \dfrac{\mathrm{d}f_z}{\mathrm{d}t} \symbfit{\hat{k}} + \big[\symbf{\Omega} \times \big( f_x \symbfit{\hat{i}} + f_y \symbfit{\hat{j}} + f_z \symbfit{\hat{k}}\big) \big] \\
			&= \Big(\dfrac{\mathrm{d}\symbfit{f}}{\mathrm{d}t}\Big)_r + \symbf{\Omega} \times \symbfit{f}(t)
			\end{aligned}
		\end{equation}
	!
		\begin{equation}
			\begin{dcases}
			\oint_l \symbfit{H} \cdot \mathrm{d}\symbfit{l} = \iint_S \symbfit{J} \cdot \mathrm{d}\symbfit{S} + \iint_S \dfrac{\partial\symbfit{D}}{\partial t} \cdot \mathrm{d}\symbfit{S} \\
			\oint_l \symbfit{E} \cdot \mathrm{d}\symbfit{l} = - \iint_S \dfrac{\partial\symbfit{B}}{\partial t} \cdot \mathrm{d}\symbfit{S} \\
			\oint_S \symbfit{B} \cdot \mathrm{d}\symbfit{S} = 0 \\
			\oint_S \symbfit{D} \cdot \mathrm{d}\symbfit{S} = \iiint_V \rho \mathrm{d}V
			\end{dcases}
		\end{equation}
	单位矩阵
		\[
		\begin{bmatrix}
		E & 0 \\
		0 & E \\
		\end{bmatrix}
		\begin{pmatrix}
			E & 0 \\
			0 & E \\
		\end{pmatrix}
		\]
\end{frame}

\subsection{页面相关}
\begin{frame}[fragile,label={fra:Fenlan}]{分栏}
	\vspace{1ex}
	\begin{columns}
		\begin{column}[t]{0.35\linewidth}
			这里是栏一
			\begin{figure}[h]
				\begin{center}
					\includegraphics[width=.8\columnwidth]{logo_name}
					\\四川大学校徽及校名\\
					\vspace{1ex}
					\includegraphics[width=.4\columnwidth]{Fy}
					\\四川大学飞扬俱乐部
				\end{center}
			\end{figure}
		\end{column}
		\pause
		\begin{column}[t]{0.4\linewidth}
			这里是栏二
			\vspace{1ex}
			\begin{itemize}
				\item 无序列表环境示例
				\begin{enumerate}
					\item 有序列表环境示例
					\item 有序列表环境示例
					\item 有序列表环境示例
				\end{enumerate}
				\item 无序列表环境示例
				\item 无序列表环境示例
			\end{itemize}
		\end{column}
		\pause
		\begin{column}[t]{0.25\linewidth}
			这里是栏三
			\par\vskip1ex
			四川大学校训\\
			\vskip2ex
			海纳百川\\[.5ex]
			有容乃大
		\end{column}
	\end{columns}
\end{frame}

\subsection{引用}
\begin{frame}[fragile]{交叉引用}
	在Beamer中应避免过多的交叉引用, 此处编者给出了常用的引用命令及其示例.
	\begin{table}[h]
		\centering
		\caption{交叉引用命令表}
		\label{tab:JiaochaYymlb}
		\begin{tabular}{lcl}
			\toprule
			命令 & 显示项 & 示例 \\
			\midrule
			\verb|\ref|\verb|{<label>}| & 序号 & \ref{assu:BianzheBhfh} \\
			\verb|\ref*|\verb|{<label>}| & 序号 & \ref*{assu:BianzheBhfh} \\
			\verb|\nameref|\verb|{<label>}| & 标题 & \nameref{assu:BianzheBhfh} \\
			\verb|\vref|\verb|{<label>}| & 标题页码 & \vref{fra:Fenlan} \\
			\verb|\pageref|\verb|{<label>}| & 页码 & \pageref{assu:BianzheBhfh} \\
			\verb|\vpageref|\verb|{<label>}| & 页码 & \vpageref{assu:BianzheBhfh} \\
			\verb|\cref|\verb|{<label>}| & 标题 & \cref{assu:BianzheBhfh} \\
			\verb|\crefrange|\verb|{<label>}| & 范围 & \crefrange{fig:ZhangmenBtdzt}{fig:ZhangmenWsdzt} \\
			\bottomrule
		\end{tabular}
	\end{table}
\end{frame}

\begin{frame}[fragile]{参考文献相关}
	\begin{itemize}[<+-| alert@+>]
		\item 脚注\footnote{这是方法一.};\\
		\item 脚注\footnotemark.
	\end{itemize}
	\footnotetext{这是方法二.}
	虚拟偶像单篇\cite{__2020-1}, 多篇\cite{__2016,m_possibilities_2018};\\
	\begin{itemize}
		\item 虚拟偶像\footnotemark.\footfullcitetext[3]{_ugc_2018}
		\item 虚拟偶像\footfullcite[8]{__2018}.\\
	\end{itemize}
\end{frame}

% --------节: 总结与思考--------
% ----------------
\section{总结与思考}
\subsection{A}
\begin{frame}{Tcolorbox\&Animate}
	\begin{minipage}[c]{0.725\columnwidth}
		\begin{scutcbrb}{锦程梦研}
			锦程梦研主题: 锦秀红与宝石蓝为渐变底色.
		\end{scutcbrb}
		\begin{scutcbyg}{浮莲落杏}
			浮莲落杏主题: 荷叶绿与银杏黄为渐变底色.
		\end{scutcbyg}
	\end{minipage}~~~~
	\begin{minipage}[c]{0.225\columnwidth}
		\begin{tcolorbox}[enhanced,%
			size=minimal,auto outer arc,%
			width=2.1cm,octogon arc,%
			colback=red,colframe=white,colupper=white,%
			fontupper=\fontsize{7mm}{7mm}\selectfont\bfseries\sffamily,%
			halign=center,valign=center,%
			square,arc is angular,%
			borderline={0.2mm}{-1mm}{red}]
			STOP
		\end{tcolorbox}
	\end{minipage}
	\begin{center}
		\begin{animateinline}[loop]{20}%
			\multiframe{160  }{rx=0+5}{
				\begin{tikzpicture}[scale=1.5]
				\pgfmathsetmacro{\x}{4+1*sin(\rx)}	
				\coordinate  (P) at (\x,0);	
				\draw [decorate,decoration={coil,segment length= \x pt, pre length=2mm, post length=2mm
					,amplitude=2mm}] (0,0)--(P);
				\shade[ball color=black](P)circle (2mm);
				\fill [pattern = north east lines] (0,0.-0.2) -- ++ (0,0.4) -- ++ (-0.1,0) -- ++(0,-0.4) --cycle
				(0,-0.2) -- ++(0,-0.1) -- ++ (5.5,0) -- ++ (0,0.1) -- cycle;
				\draw (0,0.2) -- (0,-0.2) -- (5.5,-0.2);
				\end{tikzpicture} 	    }
		\end{animateinline}
	\end{center}
\end{frame}

\subsection{B}
\begin{frame}{Tikz}
	无
\end{frame}

% --------节: 参考文献--------
% ----------------
\section{参考文献}
\subsection{参考文献}
\begin{frame}[allowframebreaks]{文献目录}
	\nocite{*}
	\printbibliography[heading=none]
\end{frame}

% --------节: 致谢--------
% ----------------
\section{致谢}
\subsection{致谢}
\begin{frame}{致谢}
	\begin{center}
		本模板参考了Beamer, Tcolorbox等手册, 感谢宏包原作者及维护者\\[1ex]
		本模板参考了知乎, Stack Overflow等平台回答, 感谢相关问题解答者\\[1ex]
		本模板使用了开源字体——楷体: 霞鹜文楷(Github \href{https://github.com/lxgw/LxgwWenKai/}{\color{scublue}LxgwWenKai} 项目), 黑体: Source Han Sans (Github \href{https://github.com/adobe-fonts/source-han-sans/}{\color{scublue}source-han-sans} 项目), 感谢字体设计师设计的优秀字体\\[1ex]
		本模板参考了中国科学技术大学Beamer模板(Github \href{https://github.com/ysx2000/USTCBeamerSX/}{\color{scublue}USTCBeamerSX} 项目), 感谢原作者提供部分设计思路\\[1ex]
		本模板参考了清华大学Beamer模板(Github \href{https://github.com/tuna/THU-Beamer-Theme/}{\color{scublue}THU-Beamer-Theme} 项目), 中国科学技术大学Beamer模板(Github \href{https://github.com/ustctug/ustcbeamer/}{\color{scublue}ustcbeamer} 项目), 感谢原作者设计的优秀模板\\[1ex]
		
		若在使用过程中发现些许Bug, 感谢诸位理解, 在此也希望诸位能先行尝试多次编译\\[1ex]
		万分感谢诸位批评指正, 感谢诸位对模板及对制作者的支持!
	\end{center}
\end{frame}

\begin{frame}{}
	\centering
	\Huge 谢谢
\end{frame}

\end{document}

%% End of file `main.tex'.