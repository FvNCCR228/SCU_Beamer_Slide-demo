%% This is file `main.tex'.
%% Version: 2021/11/30 v1.0a (Original Version: 2021/11/30 v1.0a).
%% Auther: 1701 W.
%% 本文件为 SCU_Beamer_Slide-demo 主文件源文件.
%% !Mode:: "TeX:UTF-8"
%% !XeLaTeX & Biber
%% !使用前请阅读用户手册.

% ================ %
%      导言区      %
% ================ %
\documentclass[hyperref,UTF8,11pt]{beamer}

% --------SCU Beamer 模板宏包--------
% 提供参数 chinese, minted.
% 置入参数 chinese 则交叉引用显示为中文, 置入参数 minted 则将代码显示 listings 切换为 minted.
% ----------------
\usepackage[minted,chinese]{scubeamer}

% --------不要调用的宏包--------
% ----------------
%\usepackage{pstricks} % pstricks: 绘图. (请勿调用, 本模板中调用的 adjustbox 宏包与之冲突)
%\usepackage{geometry} % geometry: 页面设置. (请勿在 Beamer 中调用)

% --------宏包调用--------
% ----------------
%\usepackage[utf8]{inputenc}
\usepackage{transparent}
\usepackage[autoplay]{animate}
\usepackage{tikz}
\usetikzlibrary{shadings,arrows,calc,decorations.pathmorphing,patterns}
\usepackage{array}
\usepackage{multirow}
\usepackage{algorithm,algorithmic}
\usepackage{amsmath,amsfonts,amssymb} % math equations, symbols
\usepackage{mathtools}
\usepackage[english]{babel}
\usepackage{color}      % color content
\usepackage{xcolor}
\usepackage{url}        % hyperlinks
\usepackage{multicol}
\usepackage{ulem} % ulem: 添加线.
\usepackage{booktabs}
\usepackage{epstopdf,epsfig}
\hypersetup{CJKbookmarks=true}
\usepackage{cprotect}
\usepackage{makecell}
\usepackage{listings}
\usepackage{subcaption}
\usepackage{varioref,cleveref,tcolorbox,biblatex}
%\usepackage[active,tightpage]{preview}
%\PreviewEnvironment{pspicture}

% 半透明化尚未出现的内容.
\setbeamercovered{transparent}
% 导入参考文献 bib 文件.
\addbibresource[location=local]{ref.bib}
% --------newcommand 区--------
% 建议在此定义常用命令.
% ----------------
\newcommand{\fverb}[1]{\texttt{#1}}

% --------导入代码环境设置--------
% ----------------
\tcblistset

% --------封面信息输入--------
% [<in footline>], {<in title page>} 方括号内容显示在页脚, 花括号内容为全称显示在封面.
% ----------------
\title[四川大学虚拟偶像研究 | Beamer模板使用答辩]{\zihao{3} 四川大学虚拟偶像研究}
\subtitle{Beamer模板使用答辩} % subtitle 未设置页脚显示项, 请在 title 中设置.
\author[我不卷, 你才卷]{\noindent 马老卷}
\institute[Business School, Sichuan University]
{%
	\noindent Management Science\\
	\medskip
	\noindent Business School, Sichuan University\\
	\medskip
	\noindent \textit{MaLSDeDiziMaLJ@scu.edu.cn}
}
\date[2021/11/31]{\noindent 2021/11/31}

% ---------------- %
%      正文区      %
% ---------------- %
\begin{document}

% --------封面--------
% ----------------
{% 取消封面页眉, 请勿删除此部分代码.
\setbeamertemplate{headline}{}%
\makeatletter
\vspace*{-\dimexpr\headheight+\topskip\relax}%
\makeatother
\begin{frame}
	% 使用 tikz 宏包中 tikzpicture 环境移动背景图片.
	\begin{tikzpicture}[overlay,remember picture]
	\node(1)[xshift=0ex,yshift=3ex] at (current page.center){%
		\pgfuseimage{bgoftitle}};
	\end{tikzpicture}
	%\transfade 渐变
	\titlepage % 显示 titlepage.
\end{frame}
}

% --------总目录--------
% 可注释.
% ----------------
%	\begin{frame}{目录}
%		%\transfade%淡入淡出 
%		\tableofcontents % 显示目录.
%	\end{frame}

% --------节: 声明--------
% ----------------
\section{引言}
\subsection{研究现状}
\begin{frame}{关于本模板}%=0.85
	\begin{itemize}
		\item 项目地址:
		\begin{itemize}
			\item 使用前请前往下列地址中查看模板版本!
			\item Github: {\color{BSblue}\url{https://github.com/FvNCCR228/SCU_Beamer_Slide-demo}}
			\item Gitee: \color{BSblue}\url{https://gitee.com/NCCR/scu_-beamer_-slide-demo}
		\end{itemize}
		\item 谢谢支持:
		\begin{itemize}
			\item 目前本模板仅在编者的Windows 11系统上编译通过, Overleaf编译失败(不是完全失败, 我很伤心);
			\item 由于编者对\LaTeX 的了解仍处在较浅层次, 故编写的模板可能会存在不兼容、编译后版式错位等现象, 希望各位能多多理解(多次编译可能有意想不到的惊喜);
			\item 若本校的\LaTeX 大佬百忙之中能对本模板提出批评指正, 鄙人在此万分感谢各位的支持.
		\end{itemize}
		\item 联系方式:
		\begin{itemize}
			\item 编者: 522869009@qq.com
		\end{itemize}
	\end{itemize}
	感谢模板中所使用部分代码的原作者, 也感谢模板所调用宏包的诸位作者前辈.
\end{frame}

\begin{frame}{使用注意}
	\begin{itemize}
		\item<1-> \LaTeX 编辑器:
		\begin{itemize}
			\item<1-> 本地: TeX Live (推荐\href{https://mirrors.tuna.tsinghua.edu.cn/CTAN/systems/texlive/Images/}{\color{BSblue}清华大学开源软件镜像站}安装最新版)配合TeXstudio或VS Code使用. TeX Live安装时间极长, 请各位做好心理准备. 此外Apple设备IDE平台建议知乎;
			\item<1-> 在线: Overleaf平台, Slager平台 (听说过但未曾使用).
		\end{itemize}
		\item<2-> \LaTeX 相关插件:
		\begin{itemize}
			\item<2-> 表格转换: Excel2\LaTeX~(\href{https://www.ctan.org/tex-archive/support/excel2latex/}{\color{BSblue}CTAN Excel2\LaTeX});
			\item<2-> 在线公式: \href{https://www.latexlive.com/}{\color{BSblue}LaTeX公式编辑器}, \href{https://mathpix.com/}{\color{BSblue}
				Mathpix}~\&~\href{https://mathf.itewqq.cn/}{\color{BSblue}图片在线转LaTeX}.
		\end{itemize}
		\item<3-> \color{JXred}!! 编译相关:
		\begin{itemize}
			\item<3-> \color{JXred}!! 请使用UTF-8格式, 设置XeLaTeX和Biber进行编译;
			\item<3-> 在线编辑请上传整个工作文件夹, 否则会出现严重错误(Bug遍地飞);
			\item<3-> \color{JXred}!! 对\LaTeX 不熟悉的情况下, 请勿轻易改动".sty"文件(宏包文件)中代码, 也可按照文件中注释进行实验性修改(注意保留备份).
		\end{itemize}
		\item<4-> \color{JXred} 建议使用Adobe Acrobat作为PDF浏览器(Ctrl+L全屏食用效果良好).
	\end{itemize}
\end{frame}

% --------节: 研究分析--------
% ----------------
\section{研究分析}
\subsection{字}
\begin{frame}[fragile]{添加线}
	\begin{multicols}{2}
		\verb|\uline|\hfill 下划线\qquad\uline{混}\\
		\verb|\uuline|\hfill 双下划线\qquad\uuline{元}\\
		\verb|\uwave|\hfill 波浪线\qquad\uwave{形}\\
		\verb|\sout|\hfill 删除线\qquad\sout{翼}\\
		\verb|\xout|\hfill 斜删除线\qquad\xout{太}\\
		\verb|\dashuline|\hfill 虚线\qquad\dashuline{极}\\
		\verb|\dotuline|\hfill 加点\qquad\dotuline{门}
	\end{multicols}
\end{frame}

\subsection{图, 表, 代码}
\begin{frame}{图}
	\begin{figure}[h]
		\begin{subfigure}{.4\columnwidth}
			\centering
			\includegraphics%
			[width=.3\columnwidth]{stop-rd.pdf}
			\caption{白天的暂停}
		\end{subfigure}
		\quad
		\begin{subfigure}{.4\columnwidth}
			\centering
			\includegraphics%
			[width=.3\columnwidth]{stop-gn.pdf}
			\caption{晚上的暂停}
		\end{subfigure}
		\caption{掌门常用的暂停}
	\end{figure}
	\begin{figure}[h]
		\begin{minipage}[t]{.4\columnwidth}
			\centering
			\includegraphics%
			[width=.3\columnwidth]{stop-rd.pdf}
			\caption{掌门白天的暂停}
			\label{fig:ZhangmenBtdzt}
		\end{minipage}
		\quad
		\begin{minipage}[t]{.4\columnwidth}
			\centering
			\includegraphics%
			[width=.3\columnwidth]{stop-gn.pdf}
			\caption{掌门晚上的暂停}
			\label{fig:ZhangmenWsdzt}
		\end{minipage}
	\end{figure}
\end{frame}

\cprotEnv\begin{frame}
	\frametitle{表}
	表格太麻烦了, 掌门说摸摸鱼, 编者觉得不错, 丢一个三线表示例. 当然也可以看看这个手册前面部分表格的源码.
	\begin{table}[htbp]
		\centering
		\caption{一些国风音乐}
		\label{tab:YixieGfyy}
		\begin{tabular}{rlc}
			\toprule
			作曲家 & 歌名 & 门中喜欢的友人 \\
			\midrule
			李志辉 & 小桥流水人家 & 门主 \\
			林海 & 无羁(器乐版) & 初号 \\
			吕秀龄 & 逆伦 & 小初 \\
			麦振鸿 & 从来只有一个人 & 编者(假的) \\
			\bottomrule
		\end{tabular}
	\end{table}
\end{frame}

\subsection{代码环境}
\begin{frame}[fragile]{代码环境演示}
	\begin{SCUcode}[]{A welcome program.}{c}{}{codecode}
		#include <stdio.h>
		int main() {
			printf("Hello world!\n");
			return 0;
		}
	\end{SCUcode}
	\begin{SCUshow}[]{A welcome program.}{c}{}{codeshow}
		#include <stdio.h>
		int main() {
			printf("Hello world!\n");
			return 0;
		}
	\end{SCUshow}
\end{frame}

\subsection{数学}
\begin{frame}[fragile,allowframebreaks]{数学环境}
	\begin{SCUproof}{}
		请读者自证.
	\end{SCUproof}
	\begin{SCUalgorithm}{怎么写Beamer模板}{ZengmoXBeamerzs}
		\begin{algorithmic}[1]
		\REQUIRE 一点点\LaTeX 知识, 不要太信任百度
		\ENSURE 不知道怎么搞
		\STATE 问门主, 肯定不知道
		\STATE 问初号, 当然不知道
		\STATE 问小初, 还是不知道
		\RETURN 算了, 不问了, 都是不知道
		\end{algorithmic}
	\end{SCUalgorithm}
	\begin{SCUdefinition}{马老卷}{MalaoJ}
		是混元形翼太极门的打砸工, 直系上峰是马凡王, 入门改姓马, 自称老卷, 实则不卷.
	\end{SCUdefinition}
	\begin{SCUlemma}{卷王森林法则}{JuanwangSlfz}
		源自未知高校学生, 此处略.
	\end{SCUlemma}
	\begin{SCUcorollary}{狼人杀的重要性}{LangrenSdzyx}
		编者实习时听公司导师说面试有可能是趣味性游戏, 狼人杀感觉很符合, 所以玩狼人杀吧.
	\end{SCUcorollary}
	\begin{SCUremark}{}{Zhu}
		\vref{coro:LangrenSdzyx}, 只是推论, 编者瞎说的.
	\end{SCUremark}
	\begin{SCUcondition}{面试狼人杀的条件}{MianshiLrsdtj}
		\vref{coro:LangrenSdzyx}, 此推论有条件, 即真有公司面试用狼人杀.
	\end{SCUcondition}
	\begin{SCUconclusion}{爱废话的编者}{AifeiHdbz}
		由上述可知: 编者爱废话.
	\end{SCUconclusion}
	\begin{SCUassumption}{编者不会废话}{BianzheBhfh}
		我们可以假设编者不会废话, 假设成立, 编者当然不会废话.
	\end{SCUassumption}
\end{frame}

\begin{frame}[fragile,allowframebreaks]{数学公式}
	麦克斯韦分布函数$f(v) = \dfrac{\mathrm{d}N}{N\,\mathrm{d}v} = 4\pi \Big(\dfrac{\mu}{2\pi kT}\Big)^{3/2} v^2 \mathrm{exp}\Big(-\dfrac{\mu v^2}{2kT}\Big)$.\\[1ex]
	最概然速率\[v_p = \sqrt{\dfrac{2kT}{\mu}} = \sqrt{\dfrac{2RT}{M}}\]其中$R$是气体常数, $M = N_A \mu$是物质的摩尔质量.\\[1ex]
		\begin{equation*}
			\bar{v} = \int_0^\infty vf(v)\,\mathrm{d}v = \sqrt{\dfrac{8kT}{\pi\mu}} = \sqrt{\dfrac{8RT}{\pi M}}
		\end{equation*}
	方均根速率
		\begin{equation}
		v_{rms} = \Big(\int_0^\infty v^2f(v)\,\mathrm{d}v\Big)^{1/2} = \sqrt{\dfrac{3kT}{\mu}} = \sqrt{\dfrac{3RT}{M}}
		\end{equation}
	$\symbb{R}\symbbit{R}\symcal{R}\symscr{R}\symfrak{R}\symsfup{R}\symsfit{R}\symbfsf{R}\symbfup{R}\symbfit{R}\symbfcal{R}\symbfscr{R}\symbffrak{R}\symbfsfup{R}\symbfsfit{R}$\\[1ex]
	多行公式 
		\begin{multline}
			A=\lim_{n\rightarrow\infty}\Delta x\left(a^{2}+\left(a^{2}+2a\Delta x+\left(\Delta x\right)^{2}\right)\right.\label{eq:reset}\\
			+\left(a^{2}+2\cdot2a\Delta x+2^{2}\left(\Delta x\right)^{2}\right)\\
			+\left(a^{2}+2\cdot3a\Delta x+3^{2}\left(\Delta x\right)^{2}\right)\\
			+\ldots\\
			\left.+\left(a^{2}+2\cdot(n-1)a\Delta x+(n-1)^{2}\left(\Delta x\right)^{2}\right)\right)\\
			=\frac{1}{3}\left(b^{3}-a^{3}\right)
		\end{multline}\\[1ex]
	质能方程
		\begin{align}
			E=m&c^2 & &E=mc^2\\
			E=~&mc^2 & E&=mc^2 \notag \\
			E&=mc^2 & E=~&mc^2\\
			&E=mc^2 & E=m&c^2
		\end{align}
	?
		\begin{equation}
			\begin{aligned}
			\dfrac{\mathrm{d}}{\mathrm{d}t} \symbfit{f} &= \dfrac{\mathrm{d}f_x}{\mathrm{d}t} \symbfit{\hat{i}} + \dfrac{\mathrm{d}\symbfit{\hat{i}}}{\mathrm{d}t} f_x + \dfrac{\mathrm{d}f_y}{\mathrm{d}t} \symbfit{\hat{j}} + \dfrac{\mathrm{d}\symbfit{\hat{j}}}{\mathrm{d}t} f_y + \dfrac{\mathrm{d}f_z}{\mathrm{d}t} \symbfit{\hat{k}} + \dfrac{\mathrm{d}\symbfit{\hat{k}}}{\mathrm{d}t} f_z \\
			&= \dfrac{\mathrm{d}f_x}{\mathrm{d}t} \symbfit{\hat{i}} + \dfrac{\mathrm{d}f_y}{\mathrm{d}t} \symbfit{\hat{j}} + \dfrac{\mathrm{d}f_z}{\mathrm{d}t} \symbfit{\hat{k}} + \big[\symbf{\Omega} \times \big( f_x \symbfit{\hat{i}} + f_y \symbfit{\hat{j}} + f_z \symbfit{\hat{k}}\big) \big] \\
			&= \Big(\dfrac{\mathrm{d}\symbfit{f}}{\mathrm{d}t}\Big)_r + \symbf{\Omega} \times \symbfit{f}(t)
			\end{aligned}
		\end{equation}
	!
		\begin{equation}
			\begin{dcases}
			\oint_l \symbfit{H} \cdot \mathrm{d}\symbfit{l} = \iint_S \symbfit{J} \cdot \mathrm{d}\symbfit{S} + \iint_S \dfrac{\partial\symbfit{D}}{\partial t} \cdot \mathrm{d}\symbfit{S} \\
			\oint_l \symbfit{E} \cdot \mathrm{d}\symbfit{l} = - \iint_S \dfrac{\partial\symbfit{B}}{\partial t} \cdot \mathrm{d}\symbfit{S} \\
			\oint_S \symbfit{B} \cdot \mathrm{d}\symbfit{S} = 0 \\
			\oint_S \symbfit{D} \cdot \mathrm{d}\symbfit{S} = \iiint_V \rho \mathrm{d}V
			\end{dcases}
		\end{equation}
	单位矩阵
		\[
		\begin{bmatrix}
		E & 0 \\
		0 & E \\
		\end{bmatrix}
		\begin{pmatrix}
			E & 0 \\
			0 & E \\
		\end{pmatrix}
		\]
\end{frame}

\subsection{页面相关}
\begin{lrbox}{\mysavebox}
	\begin{SCUremark}{卷王森林法则}{JuanwangSlfz}
		补充自\vref{lemm:JuanwangSlfz}.\par
		\textbf{\color{JXred}基本公理:}
		\begin{enumerate}
			\item 获得研究生资格是当代大学生的第一需要;
			\item 当代大学生对研究生资格的需求不断增长和扩张, 但研究生资格总量保持不变.
		\end{enumerate}
		\textbf{\color{JXred}两大重要概念:}
		\begin{enumerate}
			\item 卷疑链: 双方无法判断对方是否正在内卷;\\
			~~“当代大学生间的善意和恶意. 善和恶这类字眼放到内卷过程中是不严谨的, 所以需要对它们的含义加以限制: 善意就是指不主动内卷和卷灭其他大学生, 恶意则相反. 这是最低的善意了吧. ”
			\begin{itemize}
				\item 一个大学生不能判断另一个大学生是善还是恶
				\item 一个大学生不能判断另一个大学生认为本大学生文明是善还是恶
				\item 一个大学生不能判断另一个大学生是否会对本大学生发起内卷
				\item 一个大学生无法判断另一个大学生对自己是善意或恶意的
				\item 一个大学生无法判断另一个大学生认为自己是善意或恶意的
				\item 一个大学生无法判断另一个大学生判断自己对他是善意或恶意的
				\item \dots\dots
			\end{itemize}	
			\item 绩点爆炸: 不同大学生绩点进步的速度和加速度几乎不可能是一致的, 弱小的大学生很可能在短时间内超越强大的大学生. 可能由内因或者外因(例如内卷的交流, 内卷的程度突然加深)引发, 继而弱小的大学生能够对强大的大学生构成内卷优势乃至内卷威胁.
		\end{enumerate}
	\end{SCUremark}
\end{lrbox}
\begin{frame}[fragile,t,allowframebreaks]{环境切割}
	\framenew{.82}
\end{frame}

\begin{frame}[fragile,label={fra:Fenlan}]{分栏}
	\vspace{1ex}
	\begin{columns}
		\begin{column}[t]{0.35\linewidth}
			这里是栏一
			\begin{figure}[h]
				\begin{center}
					\includegraphics[width=.8\columnwidth]{logo_name}
					\\四川大学校徽及校名\\
					\vspace{1ex}
					\includegraphics[width=.4\columnwidth]{Fy}
					\\四川大学飞扬俱乐部
				\end{center}
			\end{figure}
		\end{column}
		\pause
		\begin{column}[t]{0.4\linewidth}
			这里是栏二
			\vspace{1ex}
			\begin{itemize}
				\item 无序列表环境示例
				\begin{enumerate}
					\item 有序列表环境示例
					\item 有序列表环境示例
					\item 有序列表环境示例
				\end{enumerate}
				\item 无序列表环境示例
				\item 无序列表环境示例
			\end{itemize}
		\end{column}
		\pause
		\begin{column}[t]{0.25\linewidth}
			这里是栏三
			\par\vskip1ex
			四川大学校训\\
			\vskip2ex
			海纳百川\\[.5ex]
			有容乃大
		\end{column}
	\end{columns}
\end{frame}

\subsection{引用}
\begin{frame}[fragile]{交叉引用}
	在Beamer中应避免过多的交叉引用, 此处编者给出了常用的引用命令及其示例.
	\begin{table}[h]
		\centering
		\caption{交叉引用命令表}
		\label{tab:JiaochaYymlb}
		\begin{tabular}{lcl}
			\toprule
			命令 & 显示项 & 示例 \\
			\midrule
			\verb|\ref|\verb|{<label>}| & 序号 & \ref{assu:BianzheBhfh} \\
			\verb|\ref*|\verb|{<label>}| & 序号 & \ref*{assu:BianzheBhfh} \\
			\verb|\nameref|\verb|{<label>}| & 标题 & \nameref{assu:BianzheBhfh} \\
			\verb|\vref|\verb|{<label>}| & 标题页码 & \vref{fra:Fenlan} \\
			\verb|\pageref|\verb|{<label>}| & 页码 & \pageref{assu:BianzheBhfh} \\
			\verb|\vpageref|\verb|{<label>}| & 页码 & \vpageref{assu:BianzheBhfh} \\
			\verb|\cref|\verb|{<label>}| & 标题 & \cref{assu:BianzheBhfh} \\
			\verb|\crefrange|\verb|{<label>}| & 范围 & \crefrange{fig:ZhangmenBtdzt}{fig:ZhangmenWsdzt} \\
			\bottomrule
		\end{tabular}
	\end{table}
\end{frame}

\begin{frame}[fragile]{参考文献相关}
	\begin{itemize}[<+-| alert@+>]
		\item 脚注\footnote{这是方法一.};\\
		\item 脚注\footnotemark.
	\end{itemize}
	\footnotetext{这是方法二.}
	虚拟偶像单篇\cite{__2020-1}, 多篇\cite{__2016,m_possibilities_2018};\\
	\begin{itemize}
		\item 虚拟偶像\footnotemark.\footfullcitetext[3]{_ugc_2018}
		\item 虚拟偶像\footfullcite[8]{__2018}.\\
	\end{itemize}
\end{frame}

% --------节: 总结与思考--------
% ----------------
\section{总结与思考}
\subsection{A}
\begin{frame}{Tcolorbox\&Animate}
	\begin{minipage}[c]{0.725\columnwidth}
		\begin{SCUtcb1}{锦程梦研}
			锦程梦研主题: 锦秀红与宝石蓝为渐变底色.
		\end{SCUtcb1}
		\begin{SCUtcb2}{浮莲落杏}
			浮莲落杏主题: 荷叶绿与银杏黄为渐变底色.
		\end{SCUtcb2}
	\end{minipage}~~~~
	\begin{minipage}[c]{0.225\columnwidth}
		\begin{tcolorbox}[enhanced,%
			size=minimal,auto outer arc,%
			width=2.1cm,octogon arc,%
			colback=red,colframe=white,colupper=white,%
			fontupper=\fontsize{7mm}{7mm}\selectfont\bfseries\sffamily,%
			halign=center,valign=center,%
			square,arc is angular,%
			borderline={0.2mm}{-1mm}{red}]
			STOP
		\end{tcolorbox}
	\end{minipage}
	\begin{center}
		\begin{animateinline}[loop]{20}%
			\multiframe{160  }{rx=0+5}{
				\begin{tikzpicture}[scale=1.5]
				\pgfmathsetmacro{\x}{4+1*sin(\rx)}	
				\coordinate  (P) at (\x,0);	
				\draw [decorate,decoration={coil,segment length= \x pt, pre length=2mm, post length=2mm
					,amplitude=2mm}] (0,0)--(P);
				\shade[ball color=black](P)circle (2mm);
				\fill [pattern = north east lines] (0,0.-0.2) -- ++ (0,0.4) -- ++ (-0.1,0) -- ++(0,-0.4) --cycle
				(0,-0.2) -- ++(0,-0.1) -- ++ (5.5,0) -- ++ (0,0.1) -- cycle;
				\draw (0,0.2) -- (0,-0.2) -- (5.5,-0.2);
				\end{tikzpicture} 	    }
		\end{animateinline}
	\end{center}
\end{frame}

\subsection{B}
\begin{frame}{Tikz}
	\begin{minipage}[c]{0.35\linewidth}
%		\scalebox{20}{}
		\begin{figure}[h]
			\definecolor{cF9DD10}{RGB}{249,221,16}
\definecolor{cEFEFEF}{RGB}{239,239,239}
\definecolor{cFBB03B}{RGB}{251,176,59}
\definecolor{c42210B}{RGB}{66,33,11}
\definecolor{cFB943B}{RGB}{251,148,59}
\definecolor{c1A1A1A}{RGB}{26,26,26}


\begin{tikzpicture}[y=0.80pt,x=0.80pt,yscale=-1, inner sep=0pt, outer sep=0pt, scale=0.2]
\path[fill=cF9DD10] (244.7832,528.3646) .. controls (244.7832,691.7943) and
(377.2898,824.3009) .. (540.7194,824.3009) .. controls (704.1490,824.3009) and
(836.6556,691.7943) .. (836.6556,528.3646) .. controls (836.6556,364.9350) and
(704.1490,232.4284) .. (540.7194,232.4284) .. controls (377.2898,232.4284) and
(244.7832,364.9350) .. (244.7832,528.3646) -- cycle(244.7832,528.3646);
\path[fill=cEFEFEF] (837.2336,399.4705) .. controls (783.4073,333.1450) and
(650.4672,339.3585) .. (560.5159,412.3311) .. controls (550.6899,423.6743) and
(551.7737,440.6531) .. (562.8279,450.1901) .. controls (573.8822,459.7271) and
(590.7887,458.2098) .. (600.5424,446.7943) .. controls (665.5675,393.9795) and
(761.9491,389.7890) .. (801.1086,438.0521) .. controls (812.0906,448.3116) and
(829.0694,447.9503) .. (838.9676,437.3296) .. controls (849.0104,426.6366) and
(848.1434,409.7301) .. (837.2336,399.4705) -- cycle(837.2336,399.4705);
\path[fill=cFBB03B] (579.7344,458.3543) .. controls (573.0152,458.3543) and
(566.5849,455.9701) .. (561.5274,451.6351) .. controls (549.6784,441.4478) and
(548.5224,423.2408) .. (559.0709,411.0306) -- (559.2154,410.8861) --
(559.3599,410.7416) .. controls (602.5654,375.7003) and (658.3425,354.1698) ..
(712.3855,351.6410) .. controls (738.3955,350.4128) and (763.3941,353.8085) ..
(784.6356,361.3225) .. controls (807.1776,369.3423) and (825.3846,381.6970) ..
(838.6786,398.0255) .. controls (844.2419,403.2275) and (847.5654,410.4526) ..
(847.9266,418.0388) .. controls (848.3601,425.6251) and (845.6146,432.9946) ..
(840.4849,438.5578) .. controls (835.2829,444.1211) and (828.1301,447.3001) ..
(820.5439,447.3723) .. controls (812.8853,447.5168) and (805.4436,444.6268) ..
(799.8081,439.3526) -- (799.5913,439.1358) .. controls (781.3121,416.6661) and
(748.8718,404.6725) .. (710.6515,406.3343) .. controls (672.2145,407.9960) and
(632.6215,423.2408) .. (601.9152,448.0948) .. controls (596.3519,454.6696) and
(588.2599,458.3543) .. (579.7344,458.3543) -- cycle(561.8887,413.7761) ..
controls (552.9297,424.3246) and (553.8689,439.9306) .. (564.0562,448.6728) ..
controls (574.2434,457.4151) and (589.9217,456.0423) .. (599.0252,445.4938) --
(599.1697,445.3493) -- (599.3142,445.2048) .. controls (630.6707,419.7728) and
(671.1307,404.2390) .. (710.5070,402.5050) .. controls (749.9555,400.7710) and
(783.5518,413.1981) .. (802.6258,436.6793) .. controls (807.4666,441.1588) and
(813.8246,443.6153) .. (820.4716,443.5431) .. controls (826.9741,443.4708) and
(833.1876,440.7253) .. (837.6671,435.9568) .. controls (842.1466,431.1883) and
(844.4586,424.8303) .. (844.0974,418.3278) .. controls (843.7361,411.6808) and
(840.8461,405.3950) .. (835.9331,400.9155) -- (835.7886,400.6988) .. controls
(810.5733,369.5590) and (765.6338,353.1583) .. (712.6745,355.6148) .. controls
(659.3540,358.1435) and (604.4439,379.3128) .. (561.8887,413.7761) --
cycle(561.8887,413.7761);
\path[fill=cEFEFEF] (519.0444,407.4903) .. controls (465.6516,340.8035) and
(332.7115,346.1500) .. (242.3267,418.5446) .. controls (232.4284,429.8878) and
(233.3677,446.7943) .. (244.3497,456.4036) .. controls (255.3317,466.0128) and
(272.2382,464.5678) .. (282.1365,453.2968) .. controls (347.5228,400.9155) and
(443.9043,397.3030) .. (482.7749,445.8551) .. controls (493.6846,456.1868) and
(510.6634,455.9701) .. (520.6339,445.3493) .. controls (530.6766,434.7286) and
(529.9541,417.7498) .. (519.0444,407.4903) -- cycle(519.0444,407.4903);
\path[fill=cFBB03B] (261.4007,464.7123) .. controls (246.5172,464.7123) and
(234.2347,453.0801) .. (233.5122,438.1966) .. controls (233.0787,430.5381) and
(235.7519,423.0241) .. (240.8094,417.2441) -- (240.9539,417.0996) --
(241.0984,416.9551) .. controls (284.5207,382.2028) and (340.4423,361.0335) ..
(394.4853,358.8660) .. controls (420.4953,357.8545) and (445.4216,361.3225) ..
(466.6631,369.0533) .. controls (489.1329,377.2175) and (507.2676,389.6445) ..
(520.4894,406.1175) .. controls (532.0494,417.1718) and (532.7719,435.3066) ..
(522.0789,446.6498) .. controls (516.7324,452.2131) and (509.3629,455.3921) ..
(501.7044,455.3198) .. controls (494.1904,455.3198) and (486.8931,452.4298) ..
(481.4021,447.2278) -- (481.1854,447.0111) .. controls (463.0506,424.3968) and
(430.7548,412.1866) .. (392.4623,413.6316) .. controls (354.0253,415.0766) and
(314.2877,430.0323) .. (283.5092,454.6696) .. controls (278.5240,460.3773) and
(271.4434,463.9898) .. (263.8572,464.5678) .. controls (262.9902,464.7123) and
(262.1954,464.7123) .. (261.4007,464.7123) -- cycle(243.6994,419.9896) ..
controls (239.3644,424.9748) and (237.1247,431.4051) .. (237.4859,437.9798) ..
controls (237.7749,444.4823) and (240.7372,450.6236) .. (245.6502,454.8863) ..
controls (250.5632,459.1491) and (256.9934,461.2443) .. (263.4959,460.6663) ..
controls (270.1429,460.0883) and (276.2842,456.9816) .. (280.6192,451.9241) --
(280.7637,451.7796) -- (280.9082,451.6351) .. controls (312.4092,426.4198) and
(353.0138,411.1028) .. (392.3178,409.6578) .. controls (431.8386,408.2128) and
(465.2903,420.8566) .. (484.2199,444.4101) .. controls (488.9161,448.8896) and
(495.2019,451.3461) .. (501.7044,451.3461) .. controls (508.3514,451.4183) and
(514.6371,448.6728) .. (519.2611,443.9043) .. controls (528.5091,434.1506) and
(527.8589,418.4001) .. (517.7439,408.8630) -- (517.5271,408.6463) .. controls
(492.4564,377.3620) and (447.6613,360.6723) .. (394.6298,362.7675) .. controls
(341.4537,364.9350) and (286.4715,385.7430) .. (243.6994,419.9896) --
cycle(243.6994,419.9896);
\path[fill=c42210B] (262.2677,413.8483) .. controls (262.2677,424.0356) and
(267.6864,433.4281) .. (276.5010,438.5578) .. controls (285.3155,443.6153) and
(296.1530,443.6153) .. (304.9675,438.5578) .. controls (313.7820,433.5003) and
(319.2007,424.0356) .. (319.2007,413.8483) .. controls (319.2007,403.6610) and
(313.7820,394.2685) .. (304.9675,389.1388) .. controls (296.1530,384.0813) and
(285.3155,384.0813) .. (276.5010,389.1388) .. controls (267.6864,394.2685) and
(262.2677,403.6610) .. (262.2677,413.8483) --
cycle(262.2677,413.8483)(575.1104,410.4526) .. controls (575.1104,420.6398)
and (580.5292,430.0323) .. (589.3437,435.1621) .. controls (598.1582,440.2196)
and (608.9957,440.2196) .. (617.8102,435.1621) .. controls (626.6247,430.1046)
and (632.0435,420.6398) .. (632.0435,410.4526) .. controls (632.0435,400.2653)
and (626.6247,390.8728) .. (617.8102,385.7430) .. controls (608.9957,380.6855)
and (598.1582,380.6855) .. (589.3437,385.7430) .. controls (580.5292,390.8728)
and (575.1104,400.3375) .. (575.1104,410.4526) -- cycle(575.1104,410.4526);
\path[fill=cFB943B] (290.8065,480.8964) .. controls (290.8065,494.4071) and
(318.9117,505.3891) .. (353.5918,505.3891) .. controls (388.2718,505.3891) and
(416.3771,494.4071) .. (416.3771,480.8964) .. controls (416.3771,467.3856) and
(388.2718,456.4036) .. (353.5918,456.4036) .. controls (318.9117,456.4036) and
(290.8065,467.3856) .. (290.8065,480.8964) --
cycle(290.8065,480.8964)(667.3015,480.8964) .. controls (667.3015,489.6386)
and (679.2950,497.7306) .. (698.6580,502.1379) .. controls (718.0933,506.5451)
and (742.0080,506.5451) .. (761.4433,502.1379) .. controls (780.8786,497.7306)
and (792.7998,489.7109) .. (792.7998,480.8964) .. controls (792.7998,472.1541)
and (780.8063,464.0621) .. (761.4433,459.6548) .. controls (742.0803,455.2476)
and (718.0933,455.2476) .. (698.6580,459.6548) .. controls (679.2227,464.0621)
and (667.3015,472.1541) .. (667.3015,480.8964) -- cycle(667.3015,480.8964);
\path[fill=c42210B] (772.5698,566.2237) .. controls (750.3890,665.7120) and
(654.6577,740.4908) .. (539.8524,740.4908) .. controls (428.7318,740.4908) and
(335.3847,670.3360) .. (309.4470,575.6162) -- (309.0135,575.6884) .. controls
(320.8625,681.6070) and (419.7006,764.3333) .. (539.8524,764.3333) .. controls
(663.6167,764.3333) and (764.6946,676.6217) .. (771.5583,566.1514);
\path[fill=c1A1A1A] (438.1243,333.2895) .. controls (422.0848,315.9495) and
(401.4935,305.5455) .. (379.0960,305.5455) .. controls (356.3373,305.5455) and
(335.4570,316.3107) .. (319.2730,334.1565) .. controls (329.3157,300.6325) and
(352.2913,277.1512) .. (379.0960,277.1512) .. controls (405.9008,277.1512) and
(428.8763,300.6325) .. (438.9191,334.1565)(762.4548,333.2895) .. controls
(746.0540,315.9495) and (725.1015,305.5455) .. (702.1983,305.5455) .. controls
(678.9337,305.5455) and (657.6200,316.3107) .. (641.1470,334.1565) .. controls
(651.4065,300.6325) and (674.8877,277.1512) .. (702.1983,277.1512) .. controls
(729.5088,277.1512) and (753.0623,300.6325) .. (763.2496,334.1565);

\end{tikzpicture}
			\vskip-5ex
			\caption{滑稽 - 向禹}
		\end{figure}
	\end{minipage}\quad
	\begin{minipage}[c]{0.6\linewidth}
%		\scalebox{20}{}
		\begin{figure}[h]
			% Dipolar magnetic field
% Author: Cyril Langlois
%
% A 3D plot of a dipolar magnetic field similar to the Earth's one.
% The field lines are drawn in each longitude plane using the dipolar field
% equation. Like Earth's field, the north magnetic pole lies close to the south
% geographic pole and vice-versa.
%
% The code is based on the Tomasz M. Trzeciak code for stereographic drawing and the plot command.

%%% Macro's for 3D figures %%%
\newcommand\pgfmathsinandcos[3]{%
	\pgfmathsetmacro#1{sin(#3)}%
	\pgfmathsetmacro#2{cos(#3)}%
}
\newcommand\LongitudePlane[3][current plane]{%
	\pgfmathsinandcos\sinEl\cosEl{#2} % elevation
	\pgfmathsinandcos\sint\cost{#3}   % azimuth
	\tikzset{#1/.estyle={cm={\cost,\sint*\sinEl,0,\cosEl,(0,0)}}}
}
\newcommand\LatitudePlane[3][current plane]{%
	\pgfmathsinandcos\sinEl\cosEl{#2} % elevation
	\pgfmathsinandcos\sint\cost{#3}   % latitude
	\pgfmathsetmacro\yshift{\cosEl*\sint}
	\tikzset{#1/.estyle={cm={\cost,0,0,\cost*\sinEl,(0,\yshift)}}} %
}
\newcommand\DrawLongitudeCircle[2][1]{
	\LongitudePlane{\angEl}{#2}
	\tikzset{current plane/.prefix style={scale=#1}}
	% angle of "visibility"
	\pgfmathsetmacro\angVis{atan(sin(#2)*cos(\angEl)/sin(\angEl))} %
	\draw[current plane] (\angVis:1) arc (\angVis:\angVis+180:1);
	\draw[current plane,dashed] (\angVis-180:1) arc (\angVis-180:\angVis:1);
}
\newcommand\DrawLatitudeCircle[2][1]{
	\LatitudePlane{\angEl}{#2}
	\tikzset{current plane/.prefix style={scale=#1}}
	\pgfmathsetmacro\sinVis{sin(#2)/cos(#2)*sin(\angEl)/cos(\angEl)}
	% angle of "visibility"
	\pgfmathsetmacro\angVis{asin(min(1,max(\sinVis,-1)))}
	\draw[current plane] (\angVis:1) arc (\angVis:-\angVis-180:1);
	\draw[current plane,dashed] (180-\angVis:1) arc (180-\angVis:\angVis:1);
}

\begin{tikzpicture}[scale=.45,
%Option for nice arrows%
>=latex,%
inner sep=0pt,%
outer sep=2pt,%
mark coordinate/.style={outer sep=0pt,
	minimum size=3pt, fill=black,circle}%
]
%% some definitions
\def\R{0.5}       % sphere radius
\def\angEl{30}    % elevation angle
\def\angAz{-140}  % azimuth angle
\def\angPhi{-105} % longitude of point
\def\angBeta{55}  % latitude of point
\def\angGam{-190} % longitude of point
%% working planes
\pgfmathsetmacro\H{\R*cos(\angEl)}          % Distance to north pole
\LongitudePlane[xzplane]{\angEl}{\angAz}    % x-axis plane
%
\coordinate (O) at (0,0);
%
\begin{scope}[rotate around={-11.1:(0,0)},
field line/.style={color=red, smooth,
	variable=\t, samples at={0,-5,-10,...,-360}}
]
\clip[rotate around={11.1:(0,0)}] (-7,5) rectangle (7,-5);

% Computes a point on a field line given r and t
\newcommand{\fieldlinecurve}[2]{%
	{(pow(#1,2))*(3*cos(#2)+cos(3*#2))}, {(pow(#1,2))*(sin(#2)+sin(3*#2))}%
}

% Longitudinal plnaes
\foreach \u in {0,-40,...,-160}{
	\LongitudePlane[{{\u}zplane}]{\angEl}{\u}
	\foreach \r in {0.25,0.5,...,2.25} {
		\draw[{{\u}zplane}, field line]
		plot (\fieldlinecurve{\r}{\t});
	}
}
\foreach \u in {-200,-240,...,-320}{
	\LongitudePlane[{{\u}zplane}]{\angEl}{\u}
	\foreach \r in {0.25,0.5,...,2.25}{
		\draw[{{\u}zplane}, dashed, field line]
		plot (\fieldlinecurve{\r}{\t});
	}
}
% Drawing plane for the B-vectors
\LongitudePlane[bzplane]{\angEl}{0}
\foreach \r in {0.25,0.5,...,2.25}{
	\draw[bzplane, thick, field line]
	plot (\fieldlinecurve{\r}{\t});
}

\begin{scope}[bzplane, very thick, ->, >=stealth]
\draw (\fieldlinecurve{1.25}{-30}) -- +(-30:0.79cm)  node[right] {$\vec{B_{r}}$};
\draw (\fieldlinecurve{1.25}{-30}) -- +(60:0.68cm)   node[right] {$\vec{B_{\theta}}$};
\draw (\fieldlinecurve{1.25}{30})  -- +(-150:0.79cm) node[below] {$\vec{B_{r}}$};
\draw (\fieldlinecurve{1.25}{30})  -- +(120:0.68cm)  node[above] {$\vec{B_{\theta}}$};
\end{scope}
%
\begin{scope}[rotate around={11.1:(0,0)}]
\fill[ball color=white,opacity=0.3] (O) circle (\R); %3D lighting effect
\draw (O) circle (\R);
\DrawLongitudeCircle[\R]{\angAz}      % xzplane
\DrawLongitudeCircle[\R]{\angAz + 90} % vzplane
\DrawLatitudeCircle[\R]{0}            % equator
\DrawLatitudeCircle[\R]{70}           % Latitude 70
\DrawLatitudeCircle[\R]{-70}          % Latitude -70
\end{scope}
%
\coordinate[mark coordinate] (Sm) at (0, \H);
	\coordinate[mark coordinate] (Nm) at (0,-\H);
	\path[xzplane] (Nm) -- +(0,-0.75) coordinate (Nm1) node[below] {$\mathbf{N}_m$}
	(Sm) -- +(0,0.75)  coordinate (Sm1) node[above] {$\mathbf{S}_m$};
	\draw[very thick, dashed]    (Sm) -- (Nm);
	\draw[very thick]            (Sm1) -- (Sm);
	\draw[very thick,->,>=latex'](Nm) -- (Nm1);
	\end{scope}
%	\node[align=justify, text width=14cm, anchor=north west] at (-7,-5.2)
%	{Schematic Earth dipolar magnetic field. The field lines placed in the
%		page plane are drawn as thick lines, those back with dashed lines and
%		the field lines in front of the page with thin lines.};
\end{tikzpicture}
			\vskip-5ex
			\caption{Dipolar Magnetic Field - Cyril Langlois}
		\end{figure}
	\end{minipage}
\end{frame}

% --------节: 参考文献--------
% ----------------
\section{参考文献}
\subsection*{参考文献}
\begin{frame}[allowframebreaks]{文献目录}
	\nocite{*}
	\printbibliography[heading=none]
\end{frame}

% --------节: 致谢--------
% ----------------
\section{致谢}
\subsection*{致谢}
\begin{frame}{~}
	\centering
	\Huge 谢谢
\end{frame}

\end{document}

%% End of file `main.tex'.
