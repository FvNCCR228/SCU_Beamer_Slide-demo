% !TeX encoding = UTF-8
% !BIB TS-program = biber
% !TeX TS-program = xelatex
% This is file `main-en.tex'.
% Copyright (C) 2021-2022 by Linrong Wu.
% Version: 2022/04/13 v1.3b (Original Version: 2022/04/13 v1.3b).
% 本文件为 SCU_Beamer_Slide-demo 主文件源文件.
% !使用前请阅读用户手册.

% ================ %
%      导言区      %
% ================ %
\documentclass[hyperref, UTF8, CJK]{beamer}
%\special{dvipdfmx:config z 0}

% --------SCU Beamer 模板宏包--------
% ----------------
\usetheme[CodeDisplay=minted, LanguageMode=en,
	ColorDisplay=BSblue, Background=false]{scu}%, BIBMode=none

% --------不要调用的宏包--------
% ----------------
%\usepackage{geometry} % geometry: 页面设置. (请勿在 Beamer 中调用)

% --------宏包调用--------
% ----------------
\usepackage{transparent}
\usepackage{array}
\usepackage{algorithm,algorithmic}
\usepackage{amsmath,amsfonts,amssymb} % math equations, symbols
\usepackage{mathtools}
\usepackage[english]{babel}
\usepackage{color}      % color content
\usepackage{url}        % hyperlinks
\usepackage{multicol,multirow}
\usepackage{ulem} % ulem: 添加线.
\usepackage{booktabs}
\usepackage{epstopdf,epsfig}
\usepackage{cprotect}
\usepackage{makecell}
\usepackage{listings}
\usepackage{subcaption}
\usepackage{varioref,cleveref}
%\usepackage[active,tightpage]{preview}
%\PreviewEnvironment{scucode}

% --------newcommand 区--------
% 建议在此定义常用命令.
% ----------------
\newcommand{\fverb}[1]{\texttt{#1}}

% --------封面信息输入--------
% [<in footline>], {<in title page>} 方括号内容显示在页脚, 花括号内容为全称显示在封面.
% ----------------
\title[A brief example in English for SCU Beamer Theme]{\LARGE A brief example in English}
\subtitle{For SCU Beamer Theme} % subtitle 未设置页脚显示项, 请在 title 中设置.
\author[Linrong Wu]{\noindent Linrong Wu}
\institute{%
	\noindent Management Science\\
	\medskip
	\noindent Business School, Sichuan University\\
	\medskip
	\noindent \textit{linrong.wu.work@outlook.com}
}
\date{\noindent \today}

% ---------------- %
%      正文区      %
% ---------------- %
\begin{document}

% --------总目录--------
% 可注释.
% ----------------
%	\begin{frame}{目录}
%		%\transfade%淡入淡出 
%		\tableofcontents % 显示目录.
%	\end{frame}

% --------节: 介绍--------
% ----------------
\section{Introduction}
\subsection{The Project}
\begin{frame}{Info.}
	\faMailForward\enspace{\color{scublue}linrong.wu.work@outlook.com}
	\faGithub\enspace{\color{scublue}\url{https://github.com/FvNCCR228/SCU_Beamer_Slide-demo}}
\end{frame}

% --------节: 区块示例--------
% ----------------
\section{Blocks}
\subsection{Math Blocks}
\begin{frame}[fragile,allowframebreaks]{Math Blocks}
	\begin{scutheorem}{A Theorem}[theorem1]
		\begin{equation}
		\dfrac{1}{n} \sum_{k=1}^{n} X_k - \dfrac{1}{n} \sum_{k=1}^{n} E(X_k) \stackrel{\;P\;}{\longrightarrow} 0
		\end{equation}
	\end{scutheorem}
	\begin{scuproof}{}
		A proof block.
	\end{scuproof}
	\begin{scuexample}{An Example}[example1]
		An example block.
	\end{scuexample}
	\begin{scualgorithm}{An Algorithm}[algorithm1]
		\begin{algorithmic}[1]
			\REQUIRE \LaTeX{}
			\ENSURE Computer
			\STATE ST
			\STATE A
			\STATE TE
			\RETURN Beamer
		\end{algorithmic}
	\end{scualgorithm}
	\begin{scudefinition}{A Defintion}[defintion1]
		A definition block.
	\end{scudefinition}
	\begin{scuaxiom}{An Axiom}[axiom1]
		An axiom block. Reference to~\Vref{def:defintion1}
	\end{scuaxiom}
	\begin{scuproperty}{A Property}[property1]
		A property block. Reference to~\Cref{axio:axiom1}
	\end{scuproperty}
	\begin{scuproposition}{A Proposition}[proposition1]
		A proposition block. Reference to~\vref{prope:property1}
		\begin{equation}
		\Delta x \Delta p \geq \dfrac{h}{4\pi}
		\end{equation}
		其中$h$为普朗克常数.
	\end{scuproposition}
	\begin{sculemma}{A lemma}[lemma1]
		A lemma block. Reference to~\cref{propo:proposition1}
	\end{sculemma}
	\begin{scucorollary}{A Corollary}[corollary1]
		A corollary block.
	\end{scucorollary}
	\begin{scuremark}{}[remark1]
		A remark block.
	\end{scuremark}
	\begin{scucondition}{A Condition}[condition1]
		A condition block.
	\end{scucondition}
	\begin{scuconclusion}{A Conclusion}[conclusion1]
		A conclusion block.
	\end{scuconclusion}
	\begin{scuassumption}{An Assumption}[assumption]
		An assumption block.
	\end{scuassumption}
\end{frame}

\begin{frame}{A Stared Block}
\begin{scutheorem}*<1-4>[after title=(after title: Theorem)]{A Stared Theorem Block}[staredblock1]
	\begin{itemize}[<+->]
		\item One
		\item Two~\only<2>{Two}
		\item \alert<3>{Three}
		\item Four
	\end{itemize}
\end{scutheorem}
\begin{scutheorem}<2-5>[after title=(after title: Theorem)]{Another Stared Theorem Block}*[staredblock2]
	\begin{itemize}
		\item Five
		\item Six~\only<3>{Six}
		\item \alert<3>{Seven}
		\item Eight
	\end{itemize}
\end{scutheorem}
\end{frame}

\subsection{Source Code Block}
\begin{frame}[fragile]{Source Code Block}{With frame option ``fragile''}
	\onslide<2>
	\begin{scucode}{A Cpp Program.}[cppcode]{c}
		#include <iostream>
		int main()
		{
			std::cout << "Hello World!" << std::endl;
			std::cin.get();
		}
	\end{scucode}
	\onslide<1>
	\begin{scucode}{A Python Program.}[pythoncode]{python}
		for i in range(1,5):
			for j in range(1,5):
				for k in range(1,5):
					if( i != k ) and (i != j) and (j != k):
						print (i,j,k)
	\end{scucode}
\end{frame}

\begin{frame}[fragile]{A Stared Source Code Block}
	\begin{scucode*}{A Stared Block.}[staredblock3]{c}
		#include <iostream>
		int main()
		{
			std::cout << "Hello World!" << std::endl;
			std::cin.get();
		}
	\end{scucode*}
	\begin{scucode}{Another Stared Theorem Block.}*[pythoncode]{python}
		for i in range(1,5):
			for j in range(1,5):
				for k in range(1,5):
					if( i != k ) and (i != j) and (j != k):
						print (i,j,k)
	\end{scucode}
\end{frame}

\begin{frame}[fragile]{Highlight Line}
	\begin{scucode}{Highlight Line.}[Highlight1]{c}[highlightlines={2,5}]
		#include <iostream>
		int main()
		{
			std::cout << "Hello World!" << std::endl;
			std::cin.get();
		}
	\end{scucode}
	\begin{scucode}{Highlight Line.}[pythoncode]{python}[highlightlines={2-3,5}]
		for i in range(1,5):
			for j in range(1,5):
				for k in range(1,5):
					if( i != k ) and (i != j) and (j != k):
						print (i,j,k)
	\end{scucode}
\end{frame}

\begin{frame}[fragile]{\LaTeX{} Comment}{Escapeinline}
	If you wanna add comments to the back of the line, it is recommended to use the corresponding language comment directly.
	\begin{scucode}{Comment.}[Comment1]{c}
		#include <iostream>
		int main()
		{// $\pi$
			std::cout << "Hello World!" << std::endl; |# \textsf{\LaTeX{} out hEllo wOrld}|
			|\colorbox{scugreen!60}{$\sum_\pi^\phi \alpha + \Gamma$}| std::cin.get(); 
		}
	\end{scucode}
	\begin{scucode}{Comment}[Comment2]{python}'@@'
		for i in range(1,5):
			for j in range(1,5): @$\sum_\pi^\phi \alpha + \Gamma$@
				for k in range(1,5): # $\sum_\pi^\phi \alpha + \Gamma$
					if( i != k ) and (i != j) and (j != k):
						print (i,j,k)
	\end{scucode}
\end{frame}

\begin{frame}[fragile]{Overlay \& Label}{Escapeinline}
	\begin{scucode}{Comment.}[Escapeinline1]{c}
		#include <iostream>
		int main()
		{
			std::cout << "Hello World!" << std::endl; // \only<1>{Value 1}\only<2>{Value 2}
			std::cin.get();
		}
	\end{scucode}
	\begin{scucode}{Comment}[Escapeinline2]{python}[highlightlines={4}]'@@'
		for i in range(1,5):
			for j in range(1,5): 
				for k in range(1,5):
					if( i != k ) and (i != j) and (j != k): @\label{line:qg}@
						print (i,j,k)
	\end{scucode}
  Reference to Line~\ref{line:qg}, the if statement.
\end{frame}

\begin{frame}[fragile]{Source Code From File}
	\scucodeinput{Source Code From File}{c}{A cpp.cpp}
\end{frame}

% --------节: 参考文献--------
% ----------------
\section{Reference}
\subsection{Reference}
\begin{frame}[allowframebreaks]{Reference}
	\nocite{*}
	\printbibliography[heading=none]
\end{frame}

% --------节: 致谢--------
% ----------------
\section{Acknowledgement}
\subsection{Acknowledgement}
\begin{frame}
	\centering
	\Huge Thanks!
\end{frame}

\end{document}

%% End of file `main-en.tex'.