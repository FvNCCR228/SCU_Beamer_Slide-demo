% !TeX encoding = UTF-8
% !BIB TS-program = biber
% !TeX TS-program = xelatex
% This is file `manual.tex'.
% Copyright (C) 2021-2024 by Linrong Wu.
% Version: 2024/10/31 v1.3e (Original Version: 2021/11/30 v1.0a).
% 本文件为 SCU_Beamer_Slide-demo 用户手册源文件.
% !使用前请阅读用户手册.

% ================ %
%      导言区      %
% ================ %
\documentclass[hyperref, UTF8, CJK, aspectratio=169]{beamer}

% --------SCU Beamer 模板宏包--------
% ----------------
\usetheme[CodeDisplay=minted, Background=true]{scu}

% --------不要调用的宏包--------
% ----------------
%\usepackage{geometry} % geometry: 页面设置. (请勿在 Beamer 中调用)

% --------宏包调用--------
% ----------------
\usepackage{transparent}
\usepackage[autoplay]{animate}
\usepackage{tikz}
\usetikzlibrary{shadings,arrows,calc,decorations.pathmorphing,patterns}
\usepackage{array}
\usepackage{algorithm,algorithmic}
\usepackage[english]{babel}
\usepackage{color}      % color content
\usepackage{url}        % hyperlinks
\usepackage{multicol,multirow}
\usepackage{ulem} % ulem: 添加线.
\usepackage{dirtree}
\usepackage{booktabs}
\usepackage{cprotect}
\usepackage{makecell}
\usepackage{listings}
\usepackage{subcaption}
\usepackage{varioref,cleveref}
%\usepackage[active,tightpage]{preview}
%\PreviewEnvironment{pspicture}

% --------newcommand 区--------
% 建议在此定义常用命令.
% ----------------
\newcommand{\fverb}[1]{\texttt{#1}}

% --------封面信息输入--------
% [<in footline>], {<in title page>} 方括号内容显示在页脚, 花括号内容为全称显示在封面.
% ----------------
\title[四川大学Beamer模板 | User's Manual Rev~v1.3e]{四川大学Beamer模板}
\subtitle{用户手册~~~~Rev v1.3e} % subtitle 未设置页脚显示项, 请在 title 中设置.
\author[我不卷, 你才卷]{马老卷\inst{1}\inst{a} \and 马小卷\inst{2}\inst{b}}
\institute{%
	\inst{1} 混元形翼太极门
	\vspace*{-6pt} \and
	\inst{2} ~Management Science, Business School, Sichuan University
	\vspace*{-6pt} \and
	\inst{a} ~\textit{MaLJFake@taichi.hunyuan} ~ \inst{b} ~\textit{MaXJFake@scu.edu.cn}
}
\date{2024年10月31日}

% ---------------- %
%      正文区      %
% ---------------- %
\begin{document}
\scriptsize
% --------总目录--------
% 可注释.
% ----------------
%	\begin{frame}{目录}
%		%\transfade%淡入淡出 
%		\tableofcontents % 显示目录.
%	\end{frame}

% --------节: 声明--------
% ----------------
\section{声明}
\subsection{编写背景}
\begin{frame}{关于Beamer}
	\begin{itemize}
		\item<1-> \TeX{}:
		\begin{itemize}
			\item<1-> 由著名的计算机科学家Donald E. Knuth发明的排版系统;
			\item<1-> 学术界中十分流行,特别在数学、物理学、统计学与计算机科学.
		\end{itemize}
		\item<2-> \LaTeX{}:
		\begin{itemize}
			\item<2-> 由L. Lamport教授开发的基于\TeX{} 的排版系统;
			\item<2-> 应用广泛, 图书、期刊、学位论文、汇报展示、简历、海报等排版;
			\item<2-> \LaTeX 相比于Word有专业的公式排版, 有大量模板降低排版难度, 有编程语言皆有的注释功能, 更能将专注度集中到文章写作中等.
		\end{itemize}
		\item<3-> Beamer:
		\begin{itemize}
			\item<3-> 一种强大而灵活的\LaTeX{} 版式, 可生成外观出色的演示文稿;
			\item<3-> 常用于学术汇报等演示.
		\end{itemize}
	\end{itemize}
\end{frame}

\begin{frame}{关于本模板}%=0.85
	\begin{itemize}
		\item 创建初衷:
		\begin{itemize}
			\item 编者本人对\LaTeX{} 稍有涉足, 这也是编者的首个Beamer模板, 模板创建源于本学院李璐老师提出的PPT修改意见;
			\item 项目也源于制作者本人的兴趣, 但本人对\LaTeX{} 的了解仍处在较浅层次, 故编写的模板可能会存在不兼容、编译后版式错位等现象;
		\end{itemize}
		\item 项目地址:
		\begin{itemize}
			\item 使用前请前往下列地址中查看模板版本!
			\item \faGithub\enspace{\color{scublue}\url{https://github.com/FvNCCR228/SCU_Beamer_Slide-demo}}
			\item Gitee: \color{scublue}\url{https://gitee.com/NCCR/SCU_Beamer_Slide-demo}
			\item main分支 - SCU Beamer Theme 核心分支\label{back:BranchMain} \scugoto{goto:BranchMain}{DTL}
			\item manual分支 - SCU Beamer Theme 用户手册分支\label{back:BranchManual} \scugoto{goto:BranchManual}{DTL}
		\end{itemize}
%			\framebreak
%		\pagebreak
		\item 联系方式:
		\begin{itemize}
			\item 制作者: lr.wu.interact@outlook.com
		\end{itemize}
	\end{itemize}
	目前仅在制作者的Windows 11 (WSL)上编译通过, Overleaf编译失败(时长问题);
\end{frame}

\begin{frame}{模板设计}
	\textbf{背景:} 封面与正文板块采用{\color{scured}不同背景}, 正文背景采用{\color{scured}低透明度淡色}, 增强正文文本等辨识性.\\~\\
	\textbf{页眉:} 采用双行设计, 首行为{\color{scured}节标题导航栏}, 显示幻灯整体思路, 还附带四川大学校名; 次行为标题栏, 左侧显示{\color{scured}小节标题}与{\color{scured}迷你帧(圆点)形式的当前小节帧进度}, 右侧显示当前{\color{scured}幻灯标题}. (编者认为小节迷你帧能在较清晰呈现进度的同时, 节约大量空间, 也能避免某节中幻灯页数过多, 导致标题导航挤压溢出)\\~\\
	\textbf{页脚:} 采用双行设计, 首行为导航栏, 左侧显示{\color{scured}报告标题}, 右侧为{\color{scured}导航模块}; 次行为信息行, 左中右分别为{\color{scured}作者}、{\color{scured}机构}、{\color{scured}日期与页码}.\\~\\
	\textbf{环境:} 模板定义了{\color{scured}定理}, {\color{scured}代码}等多种环境演示.
\end{frame}

\subsection{使用注意}
\begin{frame}{使用注意}
	\begin{itemize}
		\item<1-> \LaTeX 编辑器:
		\begin{itemize}
			\item<1-> 本地: TeX Live (推荐\href{https://mirrors.tuna.tsinghua.edu.cn/CTAN/systems/texlive/Images/}{\color{scublue}清华大学开源软件镜像站}安装最新版)配合TeXstudio或VS Code使用. TeX Live安装时间极长, 请各位做好心理准备. 此外Apple设备IDE平台建议知乎;
			\item<1-> 在线: Overleaf平台, TeXPage平台.
		\end{itemize}
		\item<2-> \LaTeX 相关插件:
		\begin{itemize}
			\item<2-> 表格转换: Excel2\LaTeX~(\href{https://www.ctan.org/tex-archive/support/excel2latex/}{\color{scublue}CTAN Excel2\LaTeX});
			\item<2-> 在线公式: \href{https://www.latexlive.com/}{\color{scublue}LaTeX公式编辑器}, \href{https://mathpix.com/}{\color{scublue}
				Mathpix}~\&~\href{https://mathf.itewqq.cn/}{\color{scublue}图片在线转LaTeX}.
		\end{itemize}
		\item<3-> \color{scured}!! 编译相关:
		\begin{itemize}
			\item<3-> \color{scured}!! 请使用UTF-8格式, 设置XeLaTeX和Biber进行编译;
			\item<3-> 在线编辑请上传整个工作文件夹, 否则会出现严重错误(Bug遍地飞);
			\item<3-> \color{scured}!! 对\LaTeX{} 不熟悉的情况下, 请勿轻易改动".sty"文件(宏包文件)中代码, 也可按照文件中注释进行实验性修改(注意保留备份).
		\end{itemize}
		\item<4-> \color{scured} 建议使用Adobe Acrobat作为PDF浏览器(Ctrl+L全屏食用效果良好).
	\end{itemize}
\end{frame}

\begin{frame}[fragile]{些许经验}
	\scriptsize
	编者设计此模板时遇到了许多问题, 以下列出部分供参考.
	\begin{table}[h]
		\centering
		\begin{tabular}{>{\raggedright\arraybackslash}p{0.88\paperwidth}}
			\hline\hline
			\centerline{\textbf{``\#''的问题(Beamer中)}}
			\textbf{报错:} \texttt{You can't use `macro parameter character \#' in xx mode.}\\
			~~a. 普通文本中(包括bibtex参考文献): 使用\verb|\#|进行转义; \\
			~~b. 使用\verb|\newcommand|等自定义命令时, 对于内部参数, 请用\verb|##1|代替\verb|#1|. (非Beamer类不用)\\
			\hline
			\centerline{\textbf{cleveref等交叉引用宏包问题(Beamer中)}}
			\textbf{在Beamer中部分标签无法正常引用及显示.}\\
			~~a. 请参考``scubeamer.sty''中Ref Layout板块, 该板块也定义了中文中的引用显示, 调用时请添加可选参数``chinese'';\\
			~~b. 此外, 请注意varioref和cleveref宏包的调用顺序.\\
			~~c. 在Beamer类文档中, \verb|\pageref|及其衍生命令仍调用PDF页码, 而非幻灯页码, 我们可通过重定义计数器page(即页码)的值来达到目标需求, 具体代码详见上述板块.\\% crefname定义不应在循环中进行.
			\hline
			\centerline{\textbf{``verb''抄录命令的问题(Beamer)}}
			\textbf{报错:} \verb|\verb| \texttt{illegal in argument.}\verb|\end|\verb!{frame}!.\\
			~~a. 请在\verb|\begin|\verb!{frame}!前加上\verb|\cprotEnv|命令, 并在导言区调用``cprotect''宏包;\\
			~~b. \verb|\begin|\verb!{frame}!后添加参数\verb|[fragile]|进行保护. 俩种方法各有优缺点.\\
			\hline
			\centerline{\textbf{``minted''高亮代码问题}}
			\textbf{报错:} \texttt{Package minted Error: You must invoke LaTeX with the...}\\
			~~需安装Python, 及Pygments组件. 并给XeLeTaX添加~-~-shell-escape参数(具体可\href{https://blog.csdn.net/weixin_39679367/article/details/111403418/}{\color{scublue}参考}). 若不使用minted, 请注释掉本文件中minted宏包调用, 以及sty文件中Listing Layout板块中指明部分. 此外, 应给XeLeTaX添加~-8bit参数消除Tab被编译为``\verb|^^I|''.\\
			\hline\hline
		\end{tabular}
	\end{table}
\end{frame}

% --------节: 基础设置--------
% ----------------
% ----------------
% 节: 基础设置
% ----------------
\section{基础设置}
\subsection{\cls{beamer} 文档类参数}
\begin{frame}{Info.}
	\textbf{本小节将介绍 \cls{beamer} 文档类部分参数.}
	\begin{multicols}{2}
		\begin{itemize}
			\item \myestablish{v1.3a}{2022/03/16}
			\item \myupdate{v1.3d}{2024/05/18}
		\end{itemize}
	\end{multicols}
	\vspace{2ex}本小节摘自\textcolor{scugreen}{Beamer User Guide}, 详细内容请在终端输入\alert{\texttt{texdoc beamer}}查看完整手册.
\end{frame}

\begin{frame}{文档类部分参数}{t\&c\&b, aspectratio}
	\textbf{调用命令}: \cmd{documentclass}\oarg*{\Arg{key1},\Arg{key2}=\Arg{value2},\ldots}\marg*{beamer}

  \textbf{调用示例}: \cmd{documentclass}\oarg*{hyperref, UTF8, CJK, aspectratio=169}\marg*{beamer}\\
	
	\alert{\texttt{t}}\&\alert{\texttt{c}}\&\alert{\texttt{b}}\hfill \textbf{Slide文本位置}
	\begin{itemize}
		\item \texttt{t}: 文本置于Slide顶部, 全局设置为\texttt{t}后允许将单个frame设置为\texttt{c}或\texttt{b}.
		\item \texttt{c}: (Default)文本置于Slide中心, 全局设置为\texttt{c}后允许将单个frame设置为\texttt{t}或\texttt{b}.
		\item 注: \texttt{b}意为将文本置于Slide底部, 为\texttt{frame}环境选项.
	\end{itemize}
	
	\vspace*{1ex}\alert{\texttt{aspectratio} = \Arg{value}}\hfill \textbf{Slide比例}
	\begin{itemize}
		\item \Arg{value} = \texttt{2013}: 比例为20:13 (W-140mm, H-91mm).
		\item \Arg{value} = \texttt{1610}: 比例为16:10 (W-160mm, H-100mm).
		\item \Arg{value} = \texttt{169}: 比例为16:9 (W-160mm, H-90mm).
		\item \Arg{value} = \texttt{149}: 比例为14:9 (W-140mm, H-90mm).
		\item \Arg{value} = \texttt{141}: 比例为1.41:1 (W-148.5mm, H-105mm).
		\item \Arg{value} = \texttt{54}: 比例为5:4 (W-125mm, H-100mm).
		\item \Arg{value} = \texttt{43}: 比例为4:3 (Default, 不可选).
		\item \Arg{value} = \texttt{32}: 比例为3:2 (W-135mm, H-90mm).
	\end{itemize}
\end{frame}

\begin{frame}{文档类部分参数}{draft, handout}
	\textbf{调用命令}: \cmd{documentclass}\oarg*{\Arg{key1},\Arg{key2}=\Arg{value2},\ldots}\marg*{beamer}

  \textbf{调用示例}: \cmd{documentclass}\oarg*{hyperref, UTF8, CJK, aspectratio=169}\marg*{beamer}\\

  \alert{\texttt{draft}}\hfill \textbf{草稿模式}

	初次编译时, 使用草稿模式以隐藏部分宏包(如 \pkg{pgf}, \pkg{hyperref})及headline, footline等的显示效果能较大提高编译速度.\\

	\alert{\texttt{handout}}\hfill \textbf{讲义版本模式}

	生成无Overlay效果的讲义版本, 可缩减页数同时降低编译时间.\\
	若需实现多合一效果, 使用 \pkg{pgfpages} 宏包(注意使用此项会破坏beamer的hyperref):
	\begin{center}
		\begin{minipage}{.8\textwidth}
			\raggedright
			\cmd{usepackage}\marg*{pgfpages}\\
			\cmd{pgfpagesuselayout}\marg*{2 on 1}\oarg*{a4paper,border shrink=5mm}
		\end{minipage}
	\end{center}
\end{frame}

\subsection{\pkg{beamethemescu} 宏包参数}
\begin{frame}{Info.}
	\textbf{本小节将介绍 \pkg{beamethemescu} 宏包参数.}
	\begin{multicols}{2}
		\begin{itemize}
			\item \myestablish{v1.1a}{2021/12/30}
			\item \myupdate{v1.3a}{2022/03/16}
			\item \myupdate{v1.3b}{2022/04/13}
			\item \myupdate{v1.3d}{2024/05/18}
			\item \myupdate{v1.3e}{2024/10/31}
		\end{itemize}
	\end{multicols}
	注: 自 \textcolor{scugreen}{v1.1a (2021/12/30)} 起, 本小节已替代手册中\alert{\textbf{可注释项}}小节.\par
	\mycopyright
\end{frame}

\begin{frame}{\pkg{beamethemescu} 宏包参数键值对说明}{MathFont, ColorDisplay, CodeDisplay}
	\pkg{beamethemescu} 是基于\LaTeX{} beamer开发的四川大学beamer样式主题宏包, 可通过命令 \cmd{usetheme}\marg*{scu}调用.

	\begin{columns}[T, onlytextwidth]% https://tex.stackexchange.com/questions/7442
		\begin{column}{.5\textwidth}
			\textbf{调用命令}(无需额外调用默认键值对):\\
			\cmd{usetheme}\oarg*{\%\\
				\hspace*{1em}\Arg{key1}=\Arg{value1},\%\\
				\hspace*{1em}\Arg{key2}=\Arg{value2},\%\\
				\hspace*{1em}\ldots,\%\\
			}\marg*{scu}
		\end{column}
		\begin{column}{.5\textwidth}
			\textbf{调用示例}:\\
			\cmd{usetheme}\oarg*{\%\\
				\hspace*{1em}ColorDisplay=BSblue,\%\\
				\hspace*{1em}BlockDisplay=followtheme,\%\\
				\hspace*{1em}MathFont=XITS,\%\\
			}\marg*{scu}
		\end{column}
	\end{columns}

	\begin{table}[h]
		\centering
		\begin{tabular}{>{\raggedleft\arraybackslash}p{0.25\textwidth}p{0.68\textwidth}}
			\textbf{Initial supported ver. \alert{\Arg{key}}} & \textbf{`C' means the key here supports a custom value.}\\
			\textbf{\Arg{value}} & \textbf{`D' means the value here is the default one.}\\
			\midrule
			v1.1a \alert{\texttt{MathFont}} & \textbf{数学字体设置}\\
			(D) \texttt{LM} & Latin Modern Math字体.\\
			\texttt{XITS} & XITS Math字体.\\
			\midrule
			v1.3a \alert{\texttt{ColorDisplay}} & \textbf{主题色显示设置}\label{back:ColorDisplay} \scugoto{goto:ColorDisplay}{PREV}\\
			(D) \texttt{JXred} & 四川大学锦锈红主题色. \textcolor{scured}{\rule[-.1ex]{1.2em}{1.8ex}} (CMYK 12, 92, 95, 20)\\
			\texttt{BSblue} & 四川大学宝石蓝主题色. \textcolor{scublue}{\rule[-.1ex]{1.2em}{1.8ex}} (CMYK 100, 60, 00, 15)\\
			v1.3c \texttt{Custom} & 自定义主题色模式. (设置前请阅读手册)\\
			\midrule
			v1.1a \alert{\texttt{CodeDisplay}} & \textbf{代码高亮显示设置}\\
			(D) \texttt{listing} & listing排版引擎.\\
			\texttt{minted} & minted排版引擎. (需Python环境, 设置前请阅读手册)\\
		\end{tabular}
	\end{table}
	\vspace*{-1ex}
\end{frame}

\begin{frame}{\pkg{beamethemescu} 宏包参数键值对说明}{MintedStyle, BlockDisplay, LanguageMode}
	\vspace*{-1.6ex}
	\begin{columns}[T, onlytextwidth]% https://tex.stackexchange.com/questions/7442
		\begin{column}{.5\textwidth}
			\textbf{调用命令}(无需额外调用默认键值对):\\
			\cmd{usetheme}\oarg*{\%\\
				\hspace*{1em}\Arg{key1}=\Arg{value1},\%\\
				\hspace*{1em}\ldots,\%\\
			}\marg*{scu}
		\end{column}
		\begin{column}{.5\textwidth}
			\textbf{调用示例}:\\
			\cmd{usetheme}\oarg*{\%\\
				\hspace*{1em}ColorDisplay=BSblue,\%\\
				\hspace*{1em}MathFont=XITS,\%\\
			}\marg*{scu}
		\end{column}
	\end{columns}

	\begin{table}[h]
		\centering
		\begin{tabular}{>{\raggedleft\arraybackslash}p{0.25\textwidth}p{0.68\textwidth}}
			v1.3a (C) \alert{\texttt{MintedStyle}} & \textbf{minted样式设置(需优先设置 \texttt{\alert{CodeDisplay}=minted})}\\
			(D) \texttt{lightmode} & 亮色模式, Pygments default样式.\\
			\texttt{darkmode} & 暗色模式, Pygments rrt样式.\\
			v1.3c \Arg{custom} & 自定义值. (设置前请阅读手册)\\
			\midrule
			v1.3a \alert{\texttt{BlockDisplay}} & \textbf{区块颜色显示设置}\\
			(D) \texttt{colorful} & 彩色模式.\\
			\texttt{followtheme} & 跟随主题色.\\
			\texttt{allgrey} & 纯灰色模式.\\
			\midrule
			v1.3b \alert{\texttt{LanguageMode}} & \textbf{语言模式设置}\\
			(D) \texttt{cn} & 中文模式.\\
			\texttt{en} & 英文模式, 支持中文输入, Headline节导航栏更窄且只显示当前节.\\
		\end{tabular}
	\end{table}
	\vspace*{-2ex}
\end{frame}

\begin{frame}{\pkg{beamethemescu} 宏包参数键值对说明}{Miniframes, NavigationTool}
	\vspace*{-1.6ex}
	\begin{columns}[T, onlytextwidth]% https://tex.stackexchange.com/questions/7442
		\begin{column}{.5\textwidth}
			\textbf{调用命令}(无需额外调用默认键值对):\\
			\cmd{usetheme}\oarg*{\%\\
				\hspace*{1em}\Arg{key1}=\Arg{value1},\%\\
				\hspace*{1em}\ldots,\%\\
			}\marg*{scu}
		\end{column}
		\begin{column}{.5\textwidth}
			\textbf{调用示例}:\\
			\cmd{usetheme}\oarg*{\%\\
				\hspace*{1em}ColorDisplay=BSblue,\%\\
				\hspace*{1em}MathFont=XITS,\%\\
			}\marg*{scu}
		\end{column}
	\end{columns}

	\begin{table}[h]
		\centering
		\begin{tabular}{>{\raggedleft\arraybackslash}p{0.25\textwidth}p{0.68\textwidth}}
			v1.3c \alert{\texttt{Miniframes}} & \textbf{页眉小节迷你帧设置}\label{back:Miniframes} \scugoto{goto:Miniframes}{PREV}\\
			(D) \texttt{follow} & 迷你帧跟随页眉小节标题.\\
			\texttt{separate} & 迷你帧与页眉小节标题分离. (实际位于页眉节导航栏右下方)\\
			\texttt{negate} & 取消小节迷你帧.\\
			\midrule
			v1.3d \alert{\texttt{NavigationTool}} & \textbf{页脚导航工具栏设置}\label{back:NavigationTool} \scugoto{goto:NavigationTool}{DTL}\\
			(D) \texttt{1-2-3} & 依次显示小节及小节跳转、查询及首尾跳转、放映及历史跳转三种工具. (可调换顺序)\\
			\texttt{1} & 类似命令为 \texttt{2} 或 \texttt{3}, 工具含义见上.\\
			\texttt{1-2} & 类似命令为 \texttt{1-3} 或 \texttt{2-3}, 工具含义见上. (可调换顺序)\\
			\texttt{negate} & 取消小节迷你帧.\\
		\end{tabular}
	\end{table}
	\vspace*{-2ex}
\end{frame}

\begin{frame}{\pkg{beamethemescu} 宏包参数键值对说明}{BIBMode, BIBStyle, ContentMuticols, Background}
	\begin{columns}[T, onlytextwidth]% https://tex.stackexchange.com/questions/7442
		\begin{column}{.5\textwidth}
			\textbf{调用命令}(无需额外调用默认键值对):\\
			\cmd{usetheme}\oarg*{\%\\
				\hspace*{1em}\Arg{key1}=\Arg{value1},\%\\
				\hspace*{1em}\ldots,\%\\
			}\marg*{scu}
		\end{column}
		\begin{column}{.5\textwidth}
			\textbf{调用示例}:\\
			\cmd{usetheme}\oarg*{\%\\
				\hspace*{1em}ColorDisplay=BSblue,\%\\
				\hspace*{1em}MathFont=XITS,\%\\
			}\marg*{scu}
		\end{column}
	\end{columns}

	\begin{table}[h]
		\centering
		\begin{tabular}{>{\raggedleft\arraybackslash}p{0.25\textwidth}p{0.68\textwidth}}
			v1.3b \alert{\texttt{BIBMode}} & \textbf{参考文献引擎设置}\\
			(D) \texttt{biber} & biber引擎.\\
			\texttt{none} & 无引擎, 不输出参考文献.\\
			\midrule
			v1.3b (C) \alert{\texttt{BIBStyle}} & \textbf{参考文献样式设置(设置 \texttt{\alert{BIBMode}=none} 时无效)}\\
			(D) \texttt{biber-gb7714} & gb7714-2015样式(biber引擎).\\
			\midrule
			v1.1a \alert{\texttt{ContentMuticols}} & \textbf{目录帧双栏显示设置}\\
			(D) \texttt{true} & 是.\\
			\texttt{false} & 否.\\
			\midrule
			v1.1a \alert{\texttt{Background}} & \textbf{背景显示设置}\\
			(D) \texttt{true} & 是.\\
			\texttt{false} & 否.\\
		\end{tabular}
	\end{table}
\end{frame}

\subsection{文档信息填写}
\begin{frame}{Info.}
	\textbf{本小节将介绍 \pkg{scu} 主题样式中封面及页脚基本信息的填写.}
	\begin{multicols}{2}
		\begin{itemize}
			\item \myestablish{v1.1a}{2021/12/30}
			\item \myupdate{v1.3d}{2024/05/18}
		\end{itemize}
	\end{multicols}
	\mycopyright
\end{frame}

\begin{frame}{封面及页脚信息}
  \structure{基本使用}

  一个Beamer演示文稿的基本信息包括标题 \alert{\texttt{title}}, 子标题 \alert{\texttt{subtitle}}, 作者 \alert{\texttt{author}}, 机构 \alert{\texttt{institute}} 和时间 \alert{\texttt{date}}. 在本模板中, 可以设置\alert{标题简称}和\alert{作者简称}使其显示在页脚中. 如下是基本的调用命令及示例: 

  \begin{columns}[T, onlytextwidth]% https://tex.stackexchange.com/questions/7442
		\begin{column}{.5\textwidth}
			\textbf{调用命令}:\\
			\cmd{title}\oarg*{\Arg{short title}}\marg*{title}\\
      \cmd{subtitle}\marg*{subtitle}\\
      \cmd{author}\oarg*{\Arg{short author}}\marg*{author}\\
      \cmd{institute}\marg*{institute}\\
      \cmd{date}\marg*{date}
		\end{column}
		\begin{column}{.5\textwidth}
			\textbf{调用示例}:\\
			\cmd{title}\oarg*{五连鞭的运气要领}\marg*{马掌门讲五连鞭的运气要领}\\
      \cmd{subtitle}\marg*{混元形翼太极门弟子的必修课}\\
      \cmd{author}\oarg*{马老卷}\marg*{掌门人马老卷}\\
      \cmd{institute}\marg*{混元形翼太极门}\\
      \cmd{date}\marg*{{2020}年{11}月{15}日}
		\end{column}
	\end{columns}

  \vspace*{3ex}\structure{页脚中添加子标题}

  通过修改文档 \cmd{title} 中 \Arg{short title} 可以实现.\\
  \textbf{调用示例}: \\
  \cmd{title}\oarg*{马掌门讲五连鞭的运气要领 | 混元形翼太极门弟子的必修课}\marg*{马掌门讲五连鞭的运气要领}\\

  \structure{跟随系统时间}

  使用 \cmd{today} 命令设置时间.\\
  \textbf{调用示例}: \\
  \cmd{date}\marg*{\cmd{today}}
\end{frame}

\begin{frame}{封面及页脚信息}
  \structure{多个作者和所属机构}

  使用 \cmd{and} 命令分隔不同作者和机构并使用 \cmd{inst} 命令设置机构标签, 并使用命令 \cmd{vspace*}\marg*{-6pt} 抑制 \cmd{institute}\marg*{} 中 \cmd{and} 命令导致的多余空白.

  \begin{columns}[T, onlytextwidth]% https://tex.stackexchange.com/questions/7442
		\begin{column}{.77\textwidth}
			\textbf{调用示例}: (注意 \cmd{inst}\marg*{} 后面若为英文文本, 请如下所示加上 \alert{\textasciitilde} 符号)\\
      \cmd{author}\oarg*{掌门人, 首席大弟子}\marg*{马老卷\cmd{inst}\marg*{1} \cmd{and} 马小卷\cmd{inst}\marg*{2}}\\
      \cmd{institute}\marg*{\%\\
        \hspace*{1em}\cmd{inst}\marg*{1} 混元形翼太极门\\
        \hspace*{1em}\alert{\textasciitilde}(\cmd{textit}\marg*{MaLJFake@taichi.hunyuan})\\
        \hspace*{1em}\cmd{vspace*}\marg*{-6pt} \cmd{and}\\
        \hspace*{1em}\cmd{inst}\marg*{2} \alert{\textasciitilde}Management Science, Business School, Sichuan University\\
        \hspace*{1em}\alert{\textbackslash\textbackslash}(\cmd{textit}\marg*{MaXJFake@scu.edu.cn})\\
      }
		\end{column}
		\begin{column}{.23\textwidth}
			\begin{figure}[h]
        \centering
        \includegraphics[width=\columnwidth]{manual-sec/manual-demo/base-settings-authors-1.pdf}
      \end{figure}
		\end{column}
	\end{columns}
  \vspace*{3ex}
  \begin{columns}[T, onlytextwidth]% https://tex.stackexchange.com/questions/7442
		\begin{column}{.77\textwidth}
			\textbf{也可将机构与邮箱分开}: \\
      \cmd{author}\oarg*{掌门人, 首席大弟子}\marg*{马老卷\cmd{inst}\marg*{1}\cmd{inst}\marg*{a} \cmd{and} 马小卷\cmd{inst}\marg*{2}\cmd{inst}\marg*{b}}\\
      \cmd{institute}\marg*{\%\\
        \hspace*{1em}\cmd{inst}\marg*{1} 混元形翼太极门\\
        \hspace*{1em}\cmd{vspace*}\marg*{-6pt} \cmd{and}\\
        \hspace*{1em}\cmd{inst}\marg*{2} \alert{\textasciitilde}Management Science, Business School, Sichuan University\\
        \hspace*{1em}\cmd{vspace*}\marg*{-6pt} \cmd{and}\\
        \hspace*{1em}\cmd{inst}\marg*{a} \alert{\textasciitilde}\cmd{textit}\marg*{MaLJFake@mail} \alert{\textasciitilde}\cmd{inst}\marg*{b} \alert{\textasciitilde}\cmd{textit}\marg*{MaXJFake@mail}\\
      }
		\end{column}
		\begin{column}{.23\textwidth}
			\begin{figure}[h]
        \centering
        \includegraphics[width=\columnwidth]{manual-sec/manual-demo/base-settings-authors-2.pdf}
      \end{figure}
		\end{column}
	\end{columns}
\end{frame}

% \begin{frame}[fragile]{封面信息}
% 	\begin{scucode}[comment={%
% 			\scriptsize%
% 			{\color{scured}<text.f>}\\
% 			\quad 指该部分键入的文字在页脚中显示\\
% 			{\color{scured}<text.t>}\\
% 			\quad 指该部分键入的文字在封面页中显示\\
% 			注意设置页脚显示时, 文字长度尽量不要超出对应位置边框. 实际使用中目标为使显示内容更详细全面, 不必拘泥于模板中信息类型的限制. 如author可置于institute处, institute简写置于author处. \\
% 			此外, 页脚处的标题可设置为``标题'', ``标题+副标题'', ``副标题''等多种样式.
% 		},%
% 		listing side comment]{封面信息设置}[FengmianXx]{tex}
% 		\title[<text.f>] {\LARGE <text.t>}
% 		\subtitle{<text.t>}
% 		\author[<text.f>]{\noindent <text.t>}
% 		\institute{%
% 			\noindent <text.t>\\
% 			\medskip
% 			\noindent <text.t>\\
% 			\medskip
% 			\noindent \textit{<text.t>}
% 		}
% 		\date{\noindent <text.t>}
% 	\end{scucode}
% \end{frame}

% --------节: 初阶使用--------
% ----------------
\section{初阶使用}
\subsection{字体}
\begin{frame}{Info.}
	\textbf{本小节将介绍字体的设置, 包括字号(未完待续).}
	\begin{multicols}{2}
		\begin{itemize}
			\item Establish: \textcolor{scugreen}{v1.0a (2021/11/30)}
			\item Update: \textcolor{scugreen}{v1.3a (2022/03/16)}
      \item Update: \textcolor{scugreen}{v1.3b (2022/04/13)}
		\end{itemize}
	\end{multicols}
	\mycopyright
\end{frame}

\begin{frame}{字号}
	\textbf{设置命令}: \\
	全局: \cmd{begin}\marg{env} \cmd{cmd} \ldots \cmd{end}\marg{env}\\
	部分: \marg*{\cmd{cmd} \Arg{text}}\par\vspace{1ex}
	\begin{columns}
		\begin{column}{.72\textwidth}
			\begin{table}[h]
				\begin{tabular}{r|l}
					    \alert{命~~令} & \alert{演~~示}                               \\
					        \cmd{Huge} & \Huge{Neighbor 老王}                         \\
					        \cmd{huge} & \huge{Neighbor 老王}                         \\
					       \cmd{LARGE} & \LARGE{Neighbor 老王}                        \\
					       \cmd{Large} & \Large{Neighbor 老王}                        \\
					       \cmd{large} & \large{Neighbor 老王}                        \\
					  \cmd{normalsize} & \normalsize{\alert{Neighbor 老王}} (Default) \\
					       \cmd{small} & \small{Neighbor 老王}                        \\
					\cmd{footnotesize} & \footnotesize{Neighbor 老王}                 \\
					  \cmd{scriptsize} & \scriptsize{Neighbor 老王}                   \\
					        \cmd{tiny} & \tiny{Neighbor 老王}
				\end{tabular}
			\end{table}
		\end{column}
		\begin{column}{.18\textwidth}
			Beamer的默认字号是\alert{normalsize}. 非特殊情况下, 为保障演示效果请勿改变全局默认字号. 当部分文字有修改字号的需求时, 一般将范围设置为\\
      \alert{footnotesize - Large}.\\[2ex]
      
			老王, 形翼门自称超级英雄第二, 道上人称Neighbor, 其特异能力是形态的缩放.
		\end{column}
	\end{columns}
\end{frame}

\begin{frame}{字体}{字体设置}
  中文字体: 楷体(粗体: \textbf{黑体})\\
  英文字体: Computer Modern Bright\\
  数学字体: $Latin\ Modern\ Math$~(sty文件中有修改为XITS(与Times风格类似)的命令)\\
\end{frame}

\begin{frame}{字体}{字体强调}
  \structure{结构字体}\\
  
  \textbf{调用命令}: \cmd{structure}\textcolor{scugrey}{<\Arg{overlay}>}\marg{text}\hfill\structure{Neighbor 老王}\\
  \textbf{调用环境}:\\
  \cmd{begin}\marg{structureenv}\textcolor{scugrey}{<\Arg{overlay}>}\\
  \hspace*{1em}\Arg{environment contents}\\
  \cmd{end}\marg{structureenv}\\[2ex]
  
  \structure{重点字体}\\
  
  \textbf{调用命令}: \cmd{alert}\textcolor{scugrey}{<\Arg{overlay}>}\marg{text}\hfill\alert{Neighbor 老王}\\
  \textbf{调用环境}:\\
  \cmd{begin}\marg{alertenv}\textcolor{scugrey}{<\Arg{overlay}>}\\
  \hspace*{1em}\Arg{environment contents}\\
  \cmd{end}\marg{alertenv}\\
  
  老王使用以上两条口诀, 能迅速变为两种不同的狂暴形态.
\end{frame}

\begin{frame}{字体}{字体族, 字体形状及字体系列}
	\begin{table}[h]
		\begin{tabularx}{.96\textwidth}{>{\raggedleft\arraybackslash}p{.1\textwidth}rl>{\centering\arraybackslash}X}
		  \toprule
		  字体族 &          调用命令           & 声明命令           & 效果                                   \\
		  \midrule
		  罗马   & \cmd{textrm}\marg{text} & \cmd{rmfami1y} & \textrm{Roman font family罗马}         \\
		  无衬线 & \cmd{textsf}\marg{text} & \cmd{sffami1y} & \textrm{Sans serif font family无衬线}   \\
		  打字机 & \cmd{texttt}\marg{text} & \cmd{ttfamily} & \texttt{Typewriter font fami1y打字机}   \\
		  \bottomrule
		\end{tabularx}
		\vskip2ex
		\begin{tabularx}{.96\textwidth}{>{\raggedleft\arraybackslash}p{.1\textwidth}rl>{\centering\arraybackslash}X}
		  \toprule
		  字体形状 & 调用命令                    & 声明命令          & 效果(Computer Modern字体族)              \\
		  \midrule
		  直立     & \cmd{textup}\marg{text} & \cmd{upshape} & \textup{Upright shape直立}            \\
		  意大利   & \cmd{textit}\marg{text} & \cmd{itshape} & \textrm{Italic shape意大利}            \\
		  倾斜     & \cmd{textsl}\marg{text} & \cmd{slshape} & \textsl{Slanted shape倾斜}            \\
		  小型大写 & \cmd{textsc}\marg{text} & \cmd{scshape} & {\scshape SMALL CAPITALS SHAPE}小型大写 \\
		  \bottomrule
		\end{tabularx}
		\vskip2ex
		\begin{tabularx}{.96\textwidth}{>{\raggedleft\arraybackslash}p{.1\textwidth}rl>{\centering\arraybackslash}X}
		  \toprule
		  字体系列 & 调用命令                    & 声明命令           & 效果                                \\
		  \midrule
		  中等     & \cmd{textmd}\marg{text} & \cmd{mdseries} & \textup{Medium series中等}          \\
		  加宽加粗 & \cmd{textbf}\marg{text} & \cmd{bfseries} & \textbf{Bold extended series加宽加粗} \\
		  \bottomrule
		\end{tabularx}
	\end{table}
	注: 此处, 编者尝试了多次仍未解决小型大写字体不正常显示的问题.
\end{frame}

\begin{frame}[fragile]{添加线及换行}
	添加线:
	\begin{multicols}{2}
		\verb|\uline|\hfill 下划线\qquad\uline{混}\\
		\verb|\uuline|\hfill 双下划线\qquad\uuline{元}\\
		\verb|\uwave|\hfill 波浪线\qquad\uwave{形}\\
		\verb|\sout|\hfill 删除线\qquad\sout{翼}\\
		\verb|\xout|\hfill 斜删除线\qquad\xout{太}\\
		\verb|\dashuline|\hfill 虚线\qquad\dashuline{极}\\
		\verb|\dotuline|\hfill 加点\qquad\dotuline{门}
	\end{multicols}
	换行:
	\begin{multicols}{2}
		\begin{itemize}
			\item 两段文字间空一行
			\item 文字结束使用"\verb!\newline!"
			\item 文字结束使用"\verb!\par!"
			\item 文字结束使用"\verb!\\[<长度>]!"\\
			~~<长度>单位一般为ex\\(当前字体尺寸设置下x的高度), 该\textbf{可选}参数指定与下行间距, 可省略.
		\end{itemize}
	\end{multicols}
\end{frame}

\subsection{版式相关}
\begin{frame}[fragile]{标点符号与颜色}
	\scriptsize
	% 标点符号部分源自武汉大学学位论文模板.
	\begin{scucode}[listing side text]{标点符号}*[BiaodianFh]{tex}
		建议使用半角标点符号, 后键入一个空格. (尤其是英文书写!)\\[1ex]
		双引号由两对左单引号、右单引号构成: ``  ''. 左单引号在键盘上ESC键下方.\\[1ex]
		建议使用实心的句号, 只要书写的是自然科学类文章.
	\end{scucode}
	\begin{scucode}[listing side text]{颜色}*[YanS]{tex}
		模板依VIS手册定义了以下颜色:\\
		\textcolor{scured}{锦绣红}
		\textcolor{scugrey}{优雅灰}
		\textcolor{scublue}{宝石蓝}
		\textcolor{scugreen}{荷叶绿}
		\textcolor{scuyellow}{银杏黄}\\
		定义颜色请使用\verb!\definecolor!
	\end{scucode}
\end{frame}
	
\cprotEnv\begin{frame}
	\frametitle{长宽间距}
	\scriptsize
	\begin{multicols}{2}
		\verb|\linewidth|\hfill 当前行的宽度\\
		\verb|\columnwidth|\hfill 当前分栏的宽度\\
		\verb|\textwidth|\hfill 整个页面版芯的宽度\\
		\verb|\textheight|\hfill 整个页面版芯的高度\\
		\verb|\paperwidth|\hfill 整个页面纸张的宽度\\
		\verb|\hspace{宽度}|\hfill 水平间距为宽度\\
		\verb|\hspace*{宽度}|\hfill 不因在行首行尾被删除\\
		\verb|\vspace{高度}|\hfill 垂直间距为高度\\
		\verb|\vspace*{高度}|\hfill 不因在页首页尾被删除\\
		\verb|\hfil|\&\verb|\hfill|\hfill 填充最大水平间距\\
		\verb|\vfil|\&\verb|\vfill|\hfill 填充最大垂直间距\\
		\verb|~|\hfill 不折行空格\\
	\end{multicols}
	\vskip-2ex
	\begin{table}[h]
		\centering
		\caption{部分空格演示}
		\begin{tabular}[scale=0.8]{l|l|r|c}
			\toprule
			代码 & 效果 & 长度 & 是否需要amsmath宏包 \\
			\midrule
			\verb|a~b| & a~b & 不知道 & 否 \\
			\verb|a\quad b| & a\quad b & 1em & 否 \\
			\verb|a\qquad b| & a\qquad b & 2em & 否 \\
			\verb|a\enspace b| & a\enspace b & 0.5em & 否 \\
			\verb|a\;b| & a\;b & 5/18em & 是 \\
			\verb|a\:b| & a\:b & 4/18em & 是 \\
			\verb|a\,b|或者~\verb|a\thinspace b| & a\,b & 3/18em & 否 \\
			\verb|a\!b|或者~\verb|a\negthinspace b| & a\!b & -3/18em & 是 \\
			\bottomrule
		\end{tabular}
	\end{table}
\end{frame}

\subsection{图与表格}
\begin{frame}[fragile]{单张图片的插入}
	\begingroup
	\scriptsize
	支持格式: {\color{scured}pdf, eps, png, jpg}. 建议使用矢量图片(svg建议下载~\href{https://inkscape.org/}{\color{scublue}Inkscape}~导出pdf)
	\endgroup
	\begin{minipage}{0.56\linewidth}
		\begin{scucode}[comment={%
				\scriptsize\vskip.5ex%
				{\color{scured}<position>}\quad 浮动体摆放的位置\\
				\quad \textbf{参数}: {\color{scublue}h}-此处, {\color{scublue}t}-顶部, {\color{scublue}b}-底部, {\color{scublue}p}-独立成页, \\
				{\color{scublue}!}-决定位置时忽略限制\\
				\quad \textbf{注}: 常用{\color{scublue}h}, {\color{scublue}htbp}, {\color{scublue}htbp!}. 参数顺序不作限制\\
				{\color{scured}<keys>}\quad 限制图片大小等\\
				\quad \textbf{参数}: {\color{scublue}width=?}-宽度, {\color{scublue}height=?}-高度, \\
				{\color{scublue}scale=?}-缩放, {\color{scublue}angle=?}-逆时针旋转角度\\
				\quad \textbf{注}: 宽高不建议同时使用, 以columnwidth设置宽度\\
				{\color{scured}<file>}\quad 文件名称(不要有空格)\\
				{\color{scured}<title>}\quad 图片标题\\
				{\color{scured}<label>}\quad 交叉引用标签\\
			},%
			listing above comment]{Figure环境}[FigureHj]{tex}
				\begin{figure}[<position>]
					\centering %居中用
					\includegraphics[<keys>]{<file>}
					\caption{<title>}
					\label{<label>}
				\end{figure}
		\end{scucode}
		\end{minipage}\quad
		\begin{minipage}{0.4\linewidth}
		\begin{scucode}[righthand width=0.2\textwidth,%
			listing above text]{单图插入演示}[DantuCrys]{tex}
				\begin{figure}[h]
					\centering
					\includegraphics%
					[width=0.36\columnwidth]%
					{stop-bk.pdf}
					\caption%
					{黑色的暂停}
					\label{fig:stopbk}
				\end{figure}
		\end{scucode}
	\end{minipage}
\end{frame}

\begin{frame}[allowframebreaks,fragile]{多图插入(子图, 并列小图)}
	放不下了: PowerPoint与Excel均可导出svg格式, 再转pdf(Excel作图时请启用``不随单元格变化''), inkscape有批量转pdf的命令, 请百度.
	
	很多时候, 我们往往不只在一行中放置一张图片, 我们需要放置并排的图片以提升空间利用率, 增强观感性, 并一定程度上加强图片的关联性. 以下列出三种不同的多图插入情景(仅供参考).
	\begin{enumerate}
		\item 在figure环境中插入多个\verb|\includegraphics|, 非子图;
		\item 使用subcaption环境插入子图;\\
		\begin{scucode}[comment={%
				\scriptsize\vskip1ex%
				{\color{scured}<width>}\quad 子图宽度\\
				{\color{scured}<code>}\quad 同上页figure环境\\
			},%
			listing side comment]{Subfigure环境}[SubfigureHj]{tex}
				\begin{subfigure}{<width>}
					<code>
				\end{subfigure}
		\end{scucode}
		\item 使用minipage环境插入并排小图, 当然minipage也可插入子图.
	\end{enumerate}
	\begin{scucode}[listing side text,righthand width=.36\columnwidth]{子图示例}[ZituSl]{tex}
		\begin{figure}[h]
			\begin{subfigure}{.4\columnwidth}
				\centering
				\includegraphics%
				[width=\columnwidth]{stop-rd.pdf}
				\caption{白天的暂停}
			\end{subfigure}
			\quad
			\begin{subfigure}{.4\columnwidth}
				\centering
				\includegraphics%
				[width=\columnwidth]{stop-gn.pdf}
				\caption{晚上的暂停}
			\end{subfigure}
			\caption{掌门常用的暂停}
		\end{figure}
	\end{scucode}
	\begin{scucode}[listing side text,righthand width=.36\columnwidth]{并列小图示例}[BinlieXtsl]{tex}
		\begin{figure}[h]
			\begin{minipage}[t]{.4\columnwidth}
				\centering
				\includegraphics%
				[width=\columnwidth]{stop-rd.pdf}
				\caption{掌门白天的暂停}
				\label{fig:ZhangmenBtdzt}
			\end{minipage}
			\quad
			\begin{minipage}[t]{.4\columnwidth}
				\centering
				\includegraphics%
				[width=\columnwidth]{stop-gn.pdf}
				\caption{掌门晚上的暂停}
				\label{fig:ZhangmenWsdzt}
			\end{minipage}
		\end{figure}
	\end{scucode}
\end{frame}

\cprotEnv\begin{frame}
	\frametitle{三线表示例}
	表格太麻烦了, 掌门说摸摸鱼, 编者觉得不错, 丢一个三线表示例. 当然也可以看看这个手册前面部分表格的源码.
	\begin{table}[htbp]
		\centering
		\caption{一些国风音乐}
		\label{tab:YixieGfyy}
		\begin{tabular}{rlc}
			\toprule
			作曲家 & 歌名 & 门中喜欢的友人 \\
			\midrule
			李志辉 & 小桥流水人家 & 门主 \\
			林海 & 无羁(器乐版) & 初号 \\
			吕秀龄 & 逆伦 & 小初 \\
			麦振鸿 & 从来只有一个人 & 编者(假的) \\
			\bottomrule
		\end{tabular}
	\end{table}
	一些三线表中有用的命令: \verb|\hline|-画横线, \verb|\cline|\verb!{x-y}!-画x-y列的横线.\\
	一些强大的表格宏包: tabularx, longtable, supertabular, xtab. 还有子表(类似于图).\\
	差点忘了, 还有合并行与列, 套表等.
\end{frame}

\subsection{代码区块}
\begin{frame}{Info.}
	\textbf{本小节将介绍代码区块(含从文件读入)的设置.}
	\begin{multicols}{2}
		\begin{itemize}
			\item Establish: \textcolor{scugreen}{v1.0a (2021/11/30)}
			\item Update: \textcolor{scugreen}{v1.3a (2022/03/16)}
		\end{itemize}
	\end{multicols}
	注: 自 \textcolor{scugreen}{v1.3a (2021/03/16)} 起, 本小节将替代手册中\alert{\textbf{代码环境}}小节.\par
	\mycopyright
\end{frame}

\begin{frame}{代码区块(含从文件读入)环境及命令定义}
	本模板定义了代码区块环境 \env{scucode} 与命令 \cmd{scucodeinput} 和 \cmd{scucodeinputnocounter}.\par\vspace{1ex}
	
	\textbf{调用环境(无星号环境)}:\\
	\cmd{begin}\marg*{scucode}\oarg{tcb options}\marg{title}\textcolor{scugrey}{\Arg{switch star}}\oarg{label suffix}\marg{code language}\oarg{code options}\textcolor{scugrey}{<\Arg{escapeinside}>}\\
	\hspace*{1em}\Arg{source code}\\
	\cmd{end}\marg*{scucode}\\[1ex]
	\textbf{调用环境(带星号环境)}:\\
	\cmd{begin}\marg*{scucode*}\oarg{tcb options}\marg{title}\textcolor{scugrey}{\Arg{switch star}}\oarg{label suffix}\marg{code language}\oarg{code options}\textcolor{scugrey}{<\Arg{escapeinside}>}\\
	\hspace*{1em}\Arg{source code}\\
	\cmd{end}\marg*{scucode*}\\[1ex]
	\textbf{以上带星号环境代表区块不显示序号.}\\
	
	\textbf{调用命令(区块显示序号)}:\\
	\cmd{scucodeinput}\oarg{tcb options}\marg{title}\textcolor{scugrey}{\Arg{switch star}}\oarg{label suffix}\marg{code language}\oarg{code options}\marg{filename}\textcolor{scugrey}{<\Arg{escapeinside}>}\\[1ex]
	\textbf{调用命令(区块不显示序号)}:\\
	\cmd{scucodeinputnocounter}\oarg{tcb options}\marg{title}\textcolor{scugrey}{\Arg{switch star}}\oarg{label suffix}\marg{code language}\oarg{code options}\marg{filename}\textcolor{scugrey}{<\Arg{escapeinside}>}
\end{frame}
	
\begin{frame}{代码区块(含从文件读入)环境及命令参数}
	\alert{\Arg{tcb options}}\hfill \textbf{(可选参数) Tcolorbox参数}\\
	添加到Tcolorbox中的参数, 常用有comment, sidebyside, listing side(above) comment等.\\
	
	\alert{\Arg{title}}\hfill \textbf{(必选参数) 标题}\\
	
	\alert{\Arg{switch star}}\hfill \textbf{(可选参数) 标题前缀显示}\\
	此处默认留空无需填入; 如填入 * 号, 则区块不显示标题前缀(源码x).\\	
		
	\alert{\Arg{label suffix}}\hfill \textbf{(可选参数) 标签后缀}\\
	模板中定义的标签均为code:xx形式, 若无填入, 则对应区块无标签.\\
	
	\alert{\Arg{code language}}\hfill \textbf{(必选参数) 代码语言}\\
	使用 \pkg{minted} 作为代码高亮引擎时, 可在终端输入\texttt{pygmentize -L lexers}查看支持语言.\\
	
	\alert{\Arg{code options}}\hfill \textbf{(可选参数) 代码引擎参数}\\
	添加到minted或listing引擎中的参数, 常用有highlightlines (listing不支持)等.\\
	
	\alert{\Arg{filename}}\hfill \textbf{(必选参数) 从文件读入源码时的文件名}\\
	
	\alert{\Arg{escapeinside}}\hfill \textbf{(可选参数) 忽略定界符内输入}\\
	将定界符间内容按\LaTeX{}命令执行, 支持Beamer \cmd{only}命令, 定界符默认为`||', 此处填入后即更改定界符.\\
	
	\alert{\Arg{source code}}\hfill \textbf{详细源码}
\end{frame}

\begin{frame}[fragile]{代码环境演示}
	\onslide<2>
	\begin{scucode}{A welcome program.}[cpphelloworld]{c}
		#include <iostream>
		int main()
		{
			std::cout << "Hello World!" << std::endl;
			std::cin.get();
		}
	\end{scucode}
	\onslide<1>
	\begin{scucode}{A welcome program.}[chelloworld]{c}
		#include <stidio.h>
		int main()
		{
			printf("Hello World!");
			return 0;
		}
	\end{scucode}
\end{frame}

\subsection{定理区块}
\begin{frame}{Info.}
	\textbf{本小节将介绍定理区块的设置.}
	\begin{multicols}{2}
		\begin{itemize}
			\item Establish: \textcolor{scugreen}{v1.0a (2021/11/30)}
			\item Update: \textcolor{scugreen}{v1.3a (2022/03/16)}
		\end{itemize}
	\end{multicols}
	注: 自 \textcolor{scugreen}{v1.3a (2021/03/16)} 起, 本小节将替代手册中\alert{\textbf{数学环境}}小节.\par
	\mycopyright
\end{frame}

\begin{frame}{定理区块环境信息}
	本模板定义了如下表所示定理区块环境及对应的标签前缀
	\begin{table}[htbp!]
		\centering
		\caption{定理区块环境信息}
		\label{tab:ShuxueHjdy}
		\begin{tabular}{ccc|ccc}
			\toprule
			名称 &          环境(scuxx)        &  标签前缀  & 名称 &          环境(scuxx)         &  标签前缀  \\ \midrule
			定理 &  \color{scured}scutheorem   &  theo:     & 证明 &    \color{scured}scuproof    &   /     \\
			例   &  \color{scured}scuexample   &  exam:     & 算法 &  \color{scured}scualgorithm  &  algo:  \\
			定义 & \color{scured}scudefinition &  def:      & 公理 &    \color{scured}scuaxiom    &  axio:  \\
			性质 &  \color{scured}scuproperty  &  prope:    & 命题 & \color{scured}scuproposition &  propo: \\
			引理 &   \color{scured}sculemma    &  lemm:     & 推论 &  \color{scured}scucorollary  &  coro:  \\
			注   &   \color{scured}scuremark   &  rema:     & 条件 &  \color{scured}scucondition  &  cond:  \\
			结论 & \color{scured}scuconclusion &  conc:     & 假设 & \color{scured}scuassumption  &  assu:  \\ \bottomrule
		\end{tabular}
	\end{table}
	其中证明环境(scuproof)结尾带有证毕符号(\cmd{qed}).$\textrightarrow$\qed
\end{frame}
	
\begin{frame}{定理区块环境命令}
	\textbf{调用环境(无星号环境)}(除scuproof与scuremark):\\
	\cmd{begin}\marg{env}\textcolor{scugrey}{<\Arg{overlay}>}\oarg{tcb options}\marg{title}\textcolor{scugrey}{\Arg{switch star}}\oarg{label suffix}\\
	\hspace*{1em}\Arg{environment contents}\\
	\cmd{end}\marg{env}\\[1ex]
	\textbf{调用环境(带星号环境)}(除scuproof与scuremark):\\
	\cmd{begin}\marg{env}*\textcolor{scugrey}{<\Arg{overlay}>}\oarg{tcb options}\marg{title}\textcolor{scugrey}{\Arg{switch star}}\oarg{label suffix}\\
	\hspace*{1em}\Arg{environment contents}\\
	\cmd{end}\marg{env}\\[1ex]
	\textbf{调用环境}(scuproof与scuremark):\\
	\cmd{begin}\marg{env}\textcolor{scugrey}{<\Arg{overlay}>}\oarg{tcb options}\marg{title}\oarg{label suffix}\\
	\hspace*{1em}\Arg{environment contents}\\
	\cmd{end}\marg{env}\\[1ex]
	\textbf{以上带星号环境代表不显示序号.}
\end{frame}

\begin{frame}{定理区块环境参数}
	\alert{\Arg{env}}\hfill \textbf{环境名称}\\
	见\vref{tab:ShuxueHjdy}.\\
	
	\alert{\Arg{overlay}}\hfill \textbf{(可选参数) Beamer Overlay设置}\\
	
	\alert{\Arg{tcb options}}\hfill \textbf{(可选参数) Tcolorbox参数}\\
	添加到Tcolorbox中的参数, 常用有comment, sidebyside, listing side(above) comment等.\\
	
	\alert{\Arg{title}}\hfill \textbf{(必选参数) 标题}\\
	
	\alert{\Arg{switch star}}\hfill \textbf{(可选参数) 标题前缀显示}\\
	此处默认留空无需填入; 如填入 * 号, 则区块不显示标题前缀(定理x、结论x等).\\	
	
	\alert{\Arg{label suffix}}\hfill \textbf{(可选参数) 标签后缀}\\
	模板中定义的标签均为xx:xx形式, 若无填入, 则对应区块无标签.\\
		
	\alert{\Arg{environment contents}}\hfill \textbf{环境内容}
\end{frame}

\begin{frame}[fragile,allowframebreaks]{数学环境演示}
	\begin{scutheorem}{切比雪夫大数率}[QiebiXfdsl]
		对独立随机变量序列$\{X_k\}$, 若$E(X_k)$, $D(X_k)$都存在, $k=1,2,\cdots$, 且有常数$C$, 使得$D(X_k)\leq C$, $k=1,2,\cdots$, 则有
		\begin{equation}
		\dfrac{1}{n} \sum_{k=1}^{n} X_k - \dfrac{1}{n} \sum_{k=1}^{n} E(X_k) \stackrel{\;P\;}{\longrightarrow} 0
		\end{equation}
	\end{scutheorem}
	\begin{scuproof}{}
		请读者自证.
	\end{scuproof}
	\begin{scuexample}{形翼门的规模}[HunyuanXytjmdgm]
		本门昨天去了80个人打水, 今天去了79个人打水, 本门的规模有多大?
	\end{scuexample}
	\begin{scualgorithm}{怎么写Beamer模板}[ZengmoXBeamerzs]
		\begin{algorithmic}[1]
		\REQUIRE 一点点\LaTeX 知识, 不要太信任百度
		\ENSURE 不知道怎么搞
		\STATE 问门主, 肯定不知道
		\STATE 问初号, 当然不知道
		\STATE 问小初, 还是不知道
		\RETURN 算了, 不问了, 都是不知道
		\end{algorithmic}
	\end{scualgorithm}
	\begin{scudefinition}{马老卷}[MalaoJ]
		是形翼门的打砸工, 直系上峰是马凡王, 入门改姓马, 自称老卷, 实则不卷.
	\end{scudefinition}
	\begin{scuaxiom}{皮亚诺公理}[PiyaNgl]
		略.
	\end{scuaxiom}
	\begin{scuproperty}{刚体的性质}[GangtiDxz]
		刚体是个理想模型. 虽然理想但是还是那么难整, 进动和章动就不会了.
	\end{scuproperty}
	\begin{scuproposition}{不确定性原理}[BuqueDxyl]
		粒子的位置与动量不可同时被确定, 位置的不确定性与动量的不确定性遵守不等式
		\begin{equation}
			\Delta x \Delta p \geq \dfrac{h}{4\pi}
		\end{equation}
		其中$h$为普朗克常数.
	\end{scuproposition}
	\begin{sculemma}{卷王森林法则}[JuanwangSlfz]
		源自未知高校学生, 此处略.
	\end{sculemma}
	\begin{scucorollary}{狼人杀的重要性}[LangrenSdzyx]
		编者实习时听公司导师说面试有可能是趣味性游戏, 狼人杀感觉很符合, 所以玩狼人杀吧.
	\end{scucorollary}
	\begin{scuremark}{}[Zhu]
		\vref{coro:LangrenSdzyx}, 只是推论, 编者瞎说的.
	\end{scuremark}
	\begin{scucondition}{面试狼人杀的条件}[MianshiLrsdtj]
		\vref{coro:LangrenSdzyx}, 此推论有条件, 即真有公司面试用狼人杀.
	\end{scucondition}
	\begin{scuconclusion}{爱废话的编者}[AifeiHdbz]
		由上述可知: 编者爱废话.
	\end{scuconclusion}
	\begin{scuassumption}{编者不会废话}[BianzheBhfh]
		我们可以假设编者不会废话, 假设成立, 编者当然不会废话.
	\end{scuassumption}
\end{frame}

\subsection{数学公式排版}
\begin{frame}{Info.}
	\textbf{本小节将介绍数学公式的排版.}
	\begin{multicols}{2}
		\begin{itemize}
			\item Establish: \textcolor{scugreen}{v1.0a (2021/11/30)}
			\item Update: \textcolor{scugreen}{v1.3a (2022/03/16)}
      \item Update: \textcolor{scugreen}{v1.3b (2022/04/13)}
		\end{itemize}
	\end{multicols}
	注: 自 \textcolor{scugreen}{v1.3a (2021/03/16)} 起, 本小节已自手册中\alert{\textbf{数学环境}}小节拆出.\par
	\mycopyright
\end{frame}

\begin{frame}[fragile,allowframebreaks]{数学公式演示}
	\alert{1. 行内公式(无编号):} 使用~\verb|$|~\verb|$|~括起公式.\\
	如: 麦克斯韦分布函数$f(v) = \dfrac{\mathrm{d}N}{N\,\mathrm{d}v} = 4\pi \Big(\dfrac{\mu}{2\pi kT}\Big)^{3/2} v^2 \mathrm{exp}\Big(-\dfrac{\mu v^2}{2kT}\Big)$.\\[1ex]
	{\color{scured}2. 行间公式(无编号):} 使用~\verb|\[|~\verb|\]|~括起公式, 与之等效的是displaymath或equation*环境.\\
	如: 最概然速率\[v_p = \sqrt{\dfrac{2kT}{\mu}} = \sqrt{\dfrac{2RT}{M}}\]其中$R$是气体常数, $M = N_A \mu$是物质的摩尔质量.\\[1ex]
		\begin{equation*}
			\bar{v} = \int_0^\infty vf(v)\,\mathrm{d}v = \sqrt{\dfrac{8kT}{\pi\mu}} = \sqrt{\dfrac{8RT}{\pi M}}
		\end{equation*}
	{\color{scured}3. 行间公式(有编号):} 使用equation环境(环境内可加label标签).\\[1ex]
	如: 方均根速率
		\begin{equation}
		v_{rms} = \Big(\int_0^\infty v^2f(v)\,\mathrm{d}v\Big)^{1/2} = \sqrt{\dfrac{3kT}{\mu}} = \sqrt{\dfrac{3RT}{M}}
		\end{equation}
	上述环境均无法使用\verb!\\!换行, 且环境中无法正常实现空格(可在空格前加\verb|\|). 在数学模式中, 若想加入文字请使用\verb|\text|\verb|{}|, 使用正体请使用\verb|\mathrm|\verb|{}|.\\[1ex]
	{\color{scured}4. 多行公式(长公式折行):} 使用multline环境.
	如: 
		\begin{multline}
			A=\lim_{n\rightarrow\infty}\Delta x\left(a^{2}+\left(a^{2}+2a\Delta x+\left(\Delta x\right)^{2}\right)\right.\label{eq:reset}\\
			+\left(a^{2}+2\cdot2a\Delta x+2^{2}\left(\Delta x\right)^{2}\right)\\
			+\left(a^{2}+2\cdot3a\Delta x+3^{2}\left(\Delta x\right)^{2}\right)\\
			+\ldots\\
			\left.+\left(a^{2}+2\cdot(n-1)a\Delta x+(n-1)^{2}\left(\Delta x\right)^{2}\right)\right)\\
			=\frac{1}{3}\left(b^{3}-a^{3}\right)
		\end{multline}\\[1ex]
	\pagebreak
	{\color{scured}5. 多行公式(有编号, 若不希望编号, 使用加*号的环境):}
	\begin{itemize}
		\item 使用align环境(每行都编号, 去掉某行编号请用\verb|\notag|命令).
	\end{itemize}
	如: 质能方程
		\begin{align}
			E=m&c^2 & &E=mc^2\\
			E=~&mc^2 & E&=mc^2 \notag \\
			E&=mc^2 & E=~&mc^2\\
			&E=mc^2 & E=m&c^2
		\end{align}
	\begin{itemize}
		\item 使用aligned环境(共同编号, 需套用在equation中).
	\end{itemize}
		\begin{equation}
			\begin{aligned}
			\dfrac{\mathrm{d}}{\mathrm{d}t} \symbfit{f} &= \dfrac{\mathrm{d}f_x}{\mathrm{d}t} \symbfit{\hat{i}} + \dfrac{\mathrm{d}\symbfit{\hat{i}}}{\mathrm{d}t} f_x + \dfrac{\mathrm{d}f_y}{\mathrm{d}t} \symbfit{\hat{j}} + \dfrac{\mathrm{d}\symbfit{\hat{j}}}{\mathrm{d}t} f_y + \dfrac{\mathrm{d}f_z}{\mathrm{d}t} \symbfit{\hat{k}} + \dfrac{\mathrm{d}\symbfit{\hat{k}}}{\mathrm{d}t} f_z \\
			&= \dfrac{\mathrm{d}f_x}{\mathrm{d}t} \symbfit{\hat{i}} + \dfrac{\mathrm{d}f_y}{\mathrm{d}t} \symbfit{\hat{j}} + \dfrac{\mathrm{d}f_z}{\mathrm{d}t} \symbfit{\hat{k}} + \big[\symbf{\Omega} \times \big( f_x \symbfit{\hat{i}} + f_y \symbfit{\hat{j}} + f_z \symbfit{\hat{k}}\big) \big] \\
			&= \Big(\dfrac{\mathrm{d}\symbfit{f}}{\mathrm{d}t}\Big)_r + \symbf{\Omega} \times \symbfit{f}(t)
			\end{aligned}
		\end{equation}
	\begin{itemize}
		\item 使用dcases环境(需调用mathtools宏包). 为什么不用cases呢? 因为这个只支持行内公式, 很多时候会出现行重合现象.
	\end{itemize}
		\begin{equation}
			\begin{dcases}
			\oint_l \symbfit{H} \cdot \mathrm{d}\symbfit{l} = \iint_S \symbfit{J} \cdot \mathrm{d}\symbfit{S} + \iint_S \dfrac{\partial\symbfit{D}}{\partial t} \cdot \mathrm{d}\symbfit{S} \\
			\oint_l \symbfit{E} \cdot \mathrm{d}\symbfit{l} = - \iint_S \dfrac{\partial\symbfit{B}}{\partial t} \cdot \mathrm{d}\symbfit{S} \\
			\oint_S \symbfit{B} \cdot \mathrm{d}\symbfit{S} = 0 \\
			\oint_S \symbfit{D} \cdot \mathrm{d}\symbfit{S} = \iiint_V \rho \mathrm{d}V
			\end{dcases}
		\end{equation}
	{\color{scured}6. 矩阵:} amsmath宏包给出了6种常用的矩阵环境, 无定界符: {\color{scured}matrix}; 有定界符: {\color{scured}pmatrix}($\big(\cdots\big)$), {\color{scured}bmatrix}($\big[\cdots\big]$), {\color{scured}Bmatrix}($\bigl\{\cdots\bigr\}$), {\color{scured}vmatrix}($|\cdots|$), {\color{scured}Vmatrix}($||\cdots||)$).
	如: 单位矩阵
		\[
		\begin{bmatrix}
		E & 0 \\
		0 & E \\
		\end{bmatrix}
		\begin{pmatrix}
			E & 0 \\
			0 & E \\
		\end{pmatrix}
		\]
\end{frame}

\begin{frame}[fragile]{数学符号}
  unicode-math宏包中定义了多个数学字体命令, 如\verb|\symbb|$\symbb{R}$, \verb|\symbbit|$\symbbit{R}$, \verb|\symcal|$\symcal{R}$, \verb|\symscr|$\symscr{R}$, \verb|\symfrak|$\symfrak{R}$, \verb|\symsfup|$\symsfup{R}$, \verb|\symsfit|$\symsfit{R}$, \verb|\symbfsf|$\symbfsf{R}$, \verb|\symbfup|$\symbfup{R}$, \verb|\symbfit|$\symbfit{R}$, \verb|\symbfcal|$\symbfcal{R}$, \verb|\symbfscr|$\symbfscr{R}$, \verb|\symbffrak|$\symbffrak{R}$, \verb|\symbfsfup|$\symbfsfup{R}$, \verb|\symbfsfit|$\symbfsfit{R}$.\\
\end{frame}

\subsection{页面相关}
\cprotEnv\begin{frame}
	\frametitle{Frame环境}
	以下主要围绕frame环境展开. (编者不一定能介绍全面, 详细请移步官方文档)
	\begin{scucode}[comment={%
			\scriptsize%
			{\color{scured}<keys>}\quad frame环境选项\\
			~~常用选项: {\color{scublue}fragile}: 保护脆弱命令(如代码, 抄录环境需此项)\\
			~~{\color{scublue}allowframebreaks(=?)}: 允许内容过多时自动切帧(括号中省略即自动判断). 注: 使用\texttt{pagebreak}或\texttt{framebreak}命令可实现手动位置切帧换页, 但会在一些时候失效(不知道为什么, 得看源码)\\
			~~{\color{scublue}t}: 在页面顶部(默认居中)\\
			{\color{scured}<title>}\quad 标题\\
		},%
		listing side comment]{Frame环境}[FrameHj]{tex}
			\begin{frame}[<keys>]{<title>}
				<code>
			\end{frame}
			或
			\begin{frame}[<keys>]
				\frametitle{<title>}
				<code>
			\end{frame}
	\end{scucode}
	对于定理等板块过长导致的无法换页问题, 此处定义了命令来处理. (尽量避免单个过长的环境)
\end{frame}

\begin{frame}[t,allowframebreaks]{环境切割示例}
	\begin{scuremark}{卷王森林法则}[JuanwangSlfz]
		补充自\vref{lemm:JuanwangSlfz}.\par
		\textbf{\color{scured}基本公理:}
		\begin{enumerate}
			\item 获得研究生资格是当代大学生的第一需要;
			\item 当代大学生对研究生资格的需求不断增长和扩张, 但研究生资格总量保持不变.
		\end{enumerate}
		\textbf{\color{scured}两大重要概念:}
		\begin{enumerate}
			\item 卷疑链: 双方无法判断对方是否正在内卷;\\
			~~“当代大学生间的善意和恶意. 善和恶这类字眼放到内卷过程中是不严谨的, 所以需要对它们的含义加以限制: 善意就是指不主动内卷和卷灭其他大学生, 恶意则相反. 这是最低的善意了吧. ”
			\begin{itemize}
				\item 一个大学生不能判断另一个大学生是善还是恶
				\item 一个大学生不能判断另一个大学生认为本大学生文明是善还是恶
				\item 一个大学生不能判断另一个大学生是否会对本大学生发起内卷
				\item 一个大学生无法判断另一个大学生对自己是善意或恶意的
				\item 一个大学生无法判断另一个大学生认为自己是善意或恶意的
				\item 一个大学生无法判断另一个大学生判断自己对他是善意或恶意的
				\item \dots\dots
			\end{itemize}	
			\item 绩点爆炸: 不同大学生绩点进步的速度和加速度几乎不可能是一致的, 弱小的大学生很可能在短时间内超越强大的大学生. 可能由内因或者外因(例如内卷的交流, 内卷的程度突然加深)引发, 继而弱小的大学生能够对强大的大学生构成内卷优势乃至内卷威胁.
		\end{enumerate}
	\end{scuremark}
\end{frame}

\subsection{分栏}
\begin{frame}[fragile,label={fra:Fenlan}]{分栏}
	此处列出两种分栏方式: \\minipage环境及columns环境嵌套column环境, 其中column环境用法同minipage(见并列小图展示\vpageref{code:BinlieXtsl}), colums环境不需要加参数. 如下列出column环境和两个列表环境的显示效果.
	\vspace{1ex}
	\pause
	\begin{columns}
		\begin{column}[t]{0.35\linewidth}
			这里是栏一
			\begin{figure}[h]
				\begin{center}
					\includegraphics[width=.8\columnwidth]{logo_name}
					\\四川大学校徽及校名\\
					\vspace{1ex}
					\includegraphics[width=.4\columnwidth]{Fy}
					\\四川大学飞扬俱乐部
				\end{center}
			\end{figure}
		\end{column}
		\pause
		\begin{column}[t]{0.4\linewidth}
			这里是栏二
			\vspace{1ex}
			\begin{itemize}
				\item 无序列表环境示例
				\begin{enumerate}
					\item 有序列表环境示例
					\item 有序列表环境示例
					\item 有序列表环境示例
				\end{enumerate}
				\item 无序列表环境示例
				\item 无序列表环境示例
			\end{itemize}
		\end{column}
		\pause
		\begin{column}[t]{0.25\linewidth}
			这里是栏三
			\par\vskip1ex
			四川大学校训\\
			\vskip2ex
			海纳百川\\[.5ex]
			有容乃大
		\end{column}
	\end{columns}
\end{frame}

\subsection{交叉引用}
\begin{frame}[fragile]{交叉引用}
	在Beamer中应避免过多的交叉引用, 此处编者给出了常用的引用命令及其示例.
	\begin{table}[h]
		\centering
		\caption{交叉引用命令表}
		\label{tab:JiaochaYymlb}
		\begin{tabular}{lcl}
			\toprule
			命令 & 显示项 & 示例 \\
			\midrule
			\verb|\ref|\verb|{<label>}| & 序号 & \ref{assu:BianzheBhfh} \\
			\verb|\ref*|\verb|{<label>}| & 序号 & \ref*{assu:BianzheBhfh} \\
			\verb|\nameref|\verb|{<label>}| & 标题 & \nameref{assu:BianzheBhfh} \\
			\verb|\vref|\verb|{<label>}| & 标题页码 & \vref{fra:Fenlan} \\
			\verb|\pageref|\verb|{<label>}| & 页码 & \pageref{assu:BianzheBhfh} \\
			\verb|\vpageref|\verb|{<label>}| & 页码 & \vpageref{assu:BianzheBhfh} \\
			\verb|\cref|\verb|{<label>}| & 标题 & \cref{assu:BianzheBhfh} \\
			\verb|\crefrange|\verb|{<label>}| & 范围 & \crefrange{fig:ZhangmenBtdzt}{fig:ZhangmenWsdzt} \\
			\bottomrule
		\end{tabular}
	\end{table}
\end{frame}

\subsection{参考文献}
\begin{frame}[fragile]{参考文献相关}
	\textbf{注意:} 参考文献请采用{\color{scured}Biber}编译模式, 即整体编译思路为XeLaTeX - Biber - XeLaTeX.\\
	模板采用符合国标GB/T7714-2015的gb7714-2015参考文献格式.\\
	模板已设置了``ref.bib''为参考文献数据库, 使用时覆盖即可(当然, 实在需要请在tex文件导言区寻找命令修改).\\[1ex]
	引用文献的命令常采用\cmd{cite}\verb|{<item>}|, 如虚拟偶像单篇\cite{__2020-1}, 多篇\cite{__2016,m_possibilities_2018};\\
	此处也可视情况使用脚注形式的详细文献信息引用:
	\begin{itemize}
		\item 使用\verb|\footnotemark|计数,配合\verb|\footfullcitetext|\verb|[<num>]|\verb|{<item>}|显示, 如虚拟偶像\footnotemark.\footfullcitetext[3]{_ugc_2018}
		\item \verb|\footfullcite|\verb|[<num>]|\verb|{<item>}|, 如虚拟偶像\footfullcite[8]{__2018}.\\
		脚注最后的黑色阿拉伯数字为参考文献序号, 需自行输入, 也即上方的\verb|[<num>]|.
	\end{itemize}
\end{frame}

\subsection{杂项}
\begin{frame}[fragile]{杂项}
	\textbf{帧的多页显示.}\\
	非列表环境: \\
	\verb|\pause|命令, \verb|\onslide<x-y>|命令(在该帧的第x到y页面显示, 为空则默认首页面或尾页面). \\
	列表环境下: 
	\begin{itemize}[<+-| alert@+>]
		\item 直接在\verb|\begin|\verb!{itemize}后加!\verb![<+-| alert@+>]!命令;
		\item 或在\verb|item|后加\verb|<x-y>|命令(在该帧的第x到y页面显示, 为空则默认首页面或尾页面).
	\end{itemize}
	\textbf{脚注.}
	\begin{itemize}[<+-| alert@+>]
		\item \verb|\footnote|\verb|{<text>}|命令\footnote{这是方法一.};\\
		\item \verb|\footnotemark|命令在正文中计数, \verb|\footnotetext|\verb|{<text>}|显示脚注\footnotemark.
	\end{itemize}
	\footnotetext{这是方法二.}
\end{frame}

% --------节: 进阶使用--------
% ----------------
\section{进阶使用}
\subsection{文本框}
\begin{frame}{Tcolorbox}
	注: 本节为进阶内容, 使用较困难, 编者本人也不会(如tcolorbox, tikz的说明文档页数分别为500+, 1300+, 均为全英文文档).\newline\par
	若您掌握一定的Tcolorbox知识, 且希望能有更好的呈现效果, 您可以在宏包模板中修改Tcolorbox设置, 或自定义文本框.\\
	在模板中, 编者除定义了定理, 代码环境的Tcolorbox文本框外. 还定义了俩种渐变文本框.\\[.5ex]
	\begin{minipage}[c]{0.725\columnwidth}
		\begin{scutcbrb}{锦程梦研}
			锦程梦研主题: 锦秀红与宝石蓝为渐变底色.
		\end{scutcbrb}
		\begin{scutcbyg}{浮莲落杏}
			浮莲落杏主题: 荷叶绿与银杏黄为渐变底色.
		\end{scutcbyg}
	\end{minipage}~~~~
	\begin{minipage}[c]{0.225\columnwidth}
		\begin{tcolorbox}[enhanced,%
			size=minimal,auto outer arc,%
			width=2.1cm,octogon arc,%
			colback=red,colframe=white,colupper=white,%
			fontupper=\fontsize{7mm}{7mm}\selectfont\bfseries\sffamily,%
			halign=center,valign=center,%
			square,arc is angular,%
			borderline={0.2mm}{-1mm}{red}]
			STOP
		\end{tcolorbox}
	\end{minipage}
	\newline\par
	\color{scugreen}此处仅展示示例, 具体操作请查看宏包手册 - ``tcolorbox.pdf''.
\end{frame}

\subsection{插图}
\begin{frame}{Tikz}
	\begin{minipage}[c]{0.35\linewidth}
%		\scalebox{20}{}
		\begin{figure}[h]
			\definecolor{cF9DD10}{RGB}{249,221,16}
\definecolor{cEFEFEF}{RGB}{239,239,239}
\definecolor{cFBB03B}{RGB}{251,176,59}
\definecolor{c42210B}{RGB}{66,33,11}
\definecolor{cFB943B}{RGB}{251,148,59}
\definecolor{c1A1A1A}{RGB}{26,26,26}


\begin{tikzpicture}[y=0.80pt,x=0.80pt,yscale=-1, inner sep=0pt, outer sep=0pt, scale=0.2]
\path[fill=cF9DD10] (244.7832,528.3646) .. controls (244.7832,691.7943) and
(377.2898,824.3009) .. (540.7194,824.3009) .. controls (704.1490,824.3009) and
(836.6556,691.7943) .. (836.6556,528.3646) .. controls (836.6556,364.9350) and
(704.1490,232.4284) .. (540.7194,232.4284) .. controls (377.2898,232.4284) and
(244.7832,364.9350) .. (244.7832,528.3646) -- cycle(244.7832,528.3646);
\path[fill=cEFEFEF] (837.2336,399.4705) .. controls (783.4073,333.1450) and
(650.4672,339.3585) .. (560.5159,412.3311) .. controls (550.6899,423.6743) and
(551.7737,440.6531) .. (562.8279,450.1901) .. controls (573.8822,459.7271) and
(590.7887,458.2098) .. (600.5424,446.7943) .. controls (665.5675,393.9795) and
(761.9491,389.7890) .. (801.1086,438.0521) .. controls (812.0906,448.3116) and
(829.0694,447.9503) .. (838.9676,437.3296) .. controls (849.0104,426.6366) and
(848.1434,409.7301) .. (837.2336,399.4705) -- cycle(837.2336,399.4705);
\path[fill=cFBB03B] (579.7344,458.3543) .. controls (573.0152,458.3543) and
(566.5849,455.9701) .. (561.5274,451.6351) .. controls (549.6784,441.4478) and
(548.5224,423.2408) .. (559.0709,411.0306) -- (559.2154,410.8861) --
(559.3599,410.7416) .. controls (602.5654,375.7003) and (658.3425,354.1698) ..
(712.3855,351.6410) .. controls (738.3955,350.4128) and (763.3941,353.8085) ..
(784.6356,361.3225) .. controls (807.1776,369.3423) and (825.3846,381.6970) ..
(838.6786,398.0255) .. controls (844.2419,403.2275) and (847.5654,410.4526) ..
(847.9266,418.0388) .. controls (848.3601,425.6251) and (845.6146,432.9946) ..
(840.4849,438.5578) .. controls (835.2829,444.1211) and (828.1301,447.3001) ..
(820.5439,447.3723) .. controls (812.8853,447.5168) and (805.4436,444.6268) ..
(799.8081,439.3526) -- (799.5913,439.1358) .. controls (781.3121,416.6661) and
(748.8718,404.6725) .. (710.6515,406.3343) .. controls (672.2145,407.9960) and
(632.6215,423.2408) .. (601.9152,448.0948) .. controls (596.3519,454.6696) and
(588.2599,458.3543) .. (579.7344,458.3543) -- cycle(561.8887,413.7761) ..
controls (552.9297,424.3246) and (553.8689,439.9306) .. (564.0562,448.6728) ..
controls (574.2434,457.4151) and (589.9217,456.0423) .. (599.0252,445.4938) --
(599.1697,445.3493) -- (599.3142,445.2048) .. controls (630.6707,419.7728) and
(671.1307,404.2390) .. (710.5070,402.5050) .. controls (749.9555,400.7710) and
(783.5518,413.1981) .. (802.6258,436.6793) .. controls (807.4666,441.1588) and
(813.8246,443.6153) .. (820.4716,443.5431) .. controls (826.9741,443.4708) and
(833.1876,440.7253) .. (837.6671,435.9568) .. controls (842.1466,431.1883) and
(844.4586,424.8303) .. (844.0974,418.3278) .. controls (843.7361,411.6808) and
(840.8461,405.3950) .. (835.9331,400.9155) -- (835.7886,400.6988) .. controls
(810.5733,369.5590) and (765.6338,353.1583) .. (712.6745,355.6148) .. controls
(659.3540,358.1435) and (604.4439,379.3128) .. (561.8887,413.7761) --
cycle(561.8887,413.7761);
\path[fill=cEFEFEF] (519.0444,407.4903) .. controls (465.6516,340.8035) and
(332.7115,346.1500) .. (242.3267,418.5446) .. controls (232.4284,429.8878) and
(233.3677,446.7943) .. (244.3497,456.4036) .. controls (255.3317,466.0128) and
(272.2382,464.5678) .. (282.1365,453.2968) .. controls (347.5228,400.9155) and
(443.9043,397.3030) .. (482.7749,445.8551) .. controls (493.6846,456.1868) and
(510.6634,455.9701) .. (520.6339,445.3493) .. controls (530.6766,434.7286) and
(529.9541,417.7498) .. (519.0444,407.4903) -- cycle(519.0444,407.4903);
\path[fill=cFBB03B] (261.4007,464.7123) .. controls (246.5172,464.7123) and
(234.2347,453.0801) .. (233.5122,438.1966) .. controls (233.0787,430.5381) and
(235.7519,423.0241) .. (240.8094,417.2441) -- (240.9539,417.0996) --
(241.0984,416.9551) .. controls (284.5207,382.2028) and (340.4423,361.0335) ..
(394.4853,358.8660) .. controls (420.4953,357.8545) and (445.4216,361.3225) ..
(466.6631,369.0533) .. controls (489.1329,377.2175) and (507.2676,389.6445) ..
(520.4894,406.1175) .. controls (532.0494,417.1718) and (532.7719,435.3066) ..
(522.0789,446.6498) .. controls (516.7324,452.2131) and (509.3629,455.3921) ..
(501.7044,455.3198) .. controls (494.1904,455.3198) and (486.8931,452.4298) ..
(481.4021,447.2278) -- (481.1854,447.0111) .. controls (463.0506,424.3968) and
(430.7548,412.1866) .. (392.4623,413.6316) .. controls (354.0253,415.0766) and
(314.2877,430.0323) .. (283.5092,454.6696) .. controls (278.5240,460.3773) and
(271.4434,463.9898) .. (263.8572,464.5678) .. controls (262.9902,464.7123) and
(262.1954,464.7123) .. (261.4007,464.7123) -- cycle(243.6994,419.9896) ..
controls (239.3644,424.9748) and (237.1247,431.4051) .. (237.4859,437.9798) ..
controls (237.7749,444.4823) and (240.7372,450.6236) .. (245.6502,454.8863) ..
controls (250.5632,459.1491) and (256.9934,461.2443) .. (263.4959,460.6663) ..
controls (270.1429,460.0883) and (276.2842,456.9816) .. (280.6192,451.9241) --
(280.7637,451.7796) -- (280.9082,451.6351) .. controls (312.4092,426.4198) and
(353.0138,411.1028) .. (392.3178,409.6578) .. controls (431.8386,408.2128) and
(465.2903,420.8566) .. (484.2199,444.4101) .. controls (488.9161,448.8896) and
(495.2019,451.3461) .. (501.7044,451.3461) .. controls (508.3514,451.4183) and
(514.6371,448.6728) .. (519.2611,443.9043) .. controls (528.5091,434.1506) and
(527.8589,418.4001) .. (517.7439,408.8630) -- (517.5271,408.6463) .. controls
(492.4564,377.3620) and (447.6613,360.6723) .. (394.6298,362.7675) .. controls
(341.4537,364.9350) and (286.4715,385.7430) .. (243.6994,419.9896) --
cycle(243.6994,419.9896);
\path[fill=c42210B] (262.2677,413.8483) .. controls (262.2677,424.0356) and
(267.6864,433.4281) .. (276.5010,438.5578) .. controls (285.3155,443.6153) and
(296.1530,443.6153) .. (304.9675,438.5578) .. controls (313.7820,433.5003) and
(319.2007,424.0356) .. (319.2007,413.8483) .. controls (319.2007,403.6610) and
(313.7820,394.2685) .. (304.9675,389.1388) .. controls (296.1530,384.0813) and
(285.3155,384.0813) .. (276.5010,389.1388) .. controls (267.6864,394.2685) and
(262.2677,403.6610) .. (262.2677,413.8483) --
cycle(262.2677,413.8483)(575.1104,410.4526) .. controls (575.1104,420.6398)
and (580.5292,430.0323) .. (589.3437,435.1621) .. controls (598.1582,440.2196)
and (608.9957,440.2196) .. (617.8102,435.1621) .. controls (626.6247,430.1046)
and (632.0435,420.6398) .. (632.0435,410.4526) .. controls (632.0435,400.2653)
and (626.6247,390.8728) .. (617.8102,385.7430) .. controls (608.9957,380.6855)
and (598.1582,380.6855) .. (589.3437,385.7430) .. controls (580.5292,390.8728)
and (575.1104,400.3375) .. (575.1104,410.4526) -- cycle(575.1104,410.4526);
\path[fill=cFB943B] (290.8065,480.8964) .. controls (290.8065,494.4071) and
(318.9117,505.3891) .. (353.5918,505.3891) .. controls (388.2718,505.3891) and
(416.3771,494.4071) .. (416.3771,480.8964) .. controls (416.3771,467.3856) and
(388.2718,456.4036) .. (353.5918,456.4036) .. controls (318.9117,456.4036) and
(290.8065,467.3856) .. (290.8065,480.8964) --
cycle(290.8065,480.8964)(667.3015,480.8964) .. controls (667.3015,489.6386)
and (679.2950,497.7306) .. (698.6580,502.1379) .. controls (718.0933,506.5451)
and (742.0080,506.5451) .. (761.4433,502.1379) .. controls (780.8786,497.7306)
and (792.7998,489.7109) .. (792.7998,480.8964) .. controls (792.7998,472.1541)
and (780.8063,464.0621) .. (761.4433,459.6548) .. controls (742.0803,455.2476)
and (718.0933,455.2476) .. (698.6580,459.6548) .. controls (679.2227,464.0621)
and (667.3015,472.1541) .. (667.3015,480.8964) -- cycle(667.3015,480.8964);
\path[fill=c42210B] (772.5698,566.2237) .. controls (750.3890,665.7120) and
(654.6577,740.4908) .. (539.8524,740.4908) .. controls (428.7318,740.4908) and
(335.3847,670.3360) .. (309.4470,575.6162) -- (309.0135,575.6884) .. controls
(320.8625,681.6070) and (419.7006,764.3333) .. (539.8524,764.3333) .. controls
(663.6167,764.3333) and (764.6946,676.6217) .. (771.5583,566.1514);
\path[fill=c1A1A1A] (438.1243,333.2895) .. controls (422.0848,315.9495) and
(401.4935,305.5455) .. (379.0960,305.5455) .. controls (356.3373,305.5455) and
(335.4570,316.3107) .. (319.2730,334.1565) .. controls (329.3157,300.6325) and
(352.2913,277.1512) .. (379.0960,277.1512) .. controls (405.9008,277.1512) and
(428.8763,300.6325) .. (438.9191,334.1565)(762.4548,333.2895) .. controls
(746.0540,315.9495) and (725.1015,305.5455) .. (702.1983,305.5455) .. controls
(678.9337,305.5455) and (657.6200,316.3107) .. (641.1470,334.1565) .. controls
(651.4065,300.6325) and (674.8877,277.1512) .. (702.1983,277.1512) .. controls
(729.5088,277.1512) and (753.0623,300.6325) .. (763.2496,334.1565);

\end{tikzpicture}
			\vskip-5ex
			\caption{滑稽 - 向禹}
		\end{figure}
	\end{minipage}\quad
	\begin{minipage}[c]{0.6\linewidth}
%		\scalebox{20}{}
		\begin{figure}[h]
			% Dipolar magnetic field
% Author: Cyril Langlois
%
% A 3D plot of a dipolar magnetic field similar to the Earth's one.
% The field lines are drawn in each longitude plane using the dipolar field
% equation. Like Earth's field, the north magnetic pole lies close to the south
% geographic pole and vice-versa.
%
% The code is based on the Tomasz M. Trzeciak code for stereographic drawing and the plot command.

%%% Macro's for 3D figures %%%
\newcommand\pgfmathsinandcos[3]{%
	\pgfmathsetmacro#1{sin(#3)}%
	\pgfmathsetmacro#2{cos(#3)}%
}
\newcommand\LongitudePlane[3][current plane]{%
	\pgfmathsinandcos\sinEl\cosEl{#2} % elevation
	\pgfmathsinandcos\sint\cost{#3}   % azimuth
	\tikzset{#1/.estyle={cm={\cost,\sint*\sinEl,0,\cosEl,(0,0)}}}
}
\newcommand\LatitudePlane[3][current plane]{%
	\pgfmathsinandcos\sinEl\cosEl{#2} % elevation
	\pgfmathsinandcos\sint\cost{#3}   % latitude
	\pgfmathsetmacro\yshift{\cosEl*\sint}
	\tikzset{#1/.estyle={cm={\cost,0,0,\cost*\sinEl,(0,\yshift)}}} %
}
\newcommand\DrawLongitudeCircle[2][1]{
	\LongitudePlane{\angEl}{#2}
	\tikzset{current plane/.prefix style={scale=#1}}
	% angle of "visibility"
	\pgfmathsetmacro\angVis{atan(sin(#2)*cos(\angEl)/sin(\angEl))} %
	\draw[current plane] (\angVis:1) arc (\angVis:\angVis+180:1);
	\draw[current plane,dashed] (\angVis-180:1) arc (\angVis-180:\angVis:1);
}
\newcommand\DrawLatitudeCircle[2][1]{
	\LatitudePlane{\angEl}{#2}
	\tikzset{current plane/.prefix style={scale=#1}}
	\pgfmathsetmacro\sinVis{sin(#2)/cos(#2)*sin(\angEl)/cos(\angEl)}
	% angle of "visibility"
	\pgfmathsetmacro\angVis{asin(min(1,max(\sinVis,-1)))}
	\draw[current plane] (\angVis:1) arc (\angVis:-\angVis-180:1);
	\draw[current plane,dashed] (180-\angVis:1) arc (180-\angVis:\angVis:1);
}

\begin{tikzpicture}[scale=.45,
%Option for nice arrows%
>=latex,%
inner sep=0pt,%
outer sep=2pt,%
mark coordinate/.style={outer sep=0pt,
	minimum size=3pt, fill=black,circle}%
]
%% some definitions
\def\R{0.5}       % sphere radius
\def\angEl{30}    % elevation angle
\def\angAz{-140}  % azimuth angle
\def\angPhi{-105} % longitude of point
\def\angBeta{55}  % latitude of point
\def\angGam{-190} % longitude of point
%% working planes
\pgfmathsetmacro\H{\R*cos(\angEl)}          % Distance to north pole
\LongitudePlane[xzplane]{\angEl}{\angAz}    % x-axis plane
%
\coordinate (O) at (0,0);
%
\begin{scope}[rotate around={-11.1:(0,0)},
field line/.style={color=red, smooth,
	variable=\t, samples at={0,-5,-10,...,-360}}
]
\clip[rotate around={11.1:(0,0)}] (-7,5) rectangle (7,-5);

% Computes a point on a field line given r and t
\newcommand{\fieldlinecurve}[2]{%
	{(pow(#1,2))*(3*cos(#2)+cos(3*#2))}, {(pow(#1,2))*(sin(#2)+sin(3*#2))}%
}

% Longitudinal plnaes
\foreach \u in {0,-40,...,-160}{
	\LongitudePlane[{{\u}zplane}]{\angEl}{\u}
	\foreach \r in {0.25,0.5,...,2.25} {
		\draw[{{\u}zplane}, field line]
		plot (\fieldlinecurve{\r}{\t});
	}
}
\foreach \u in {-200,-240,...,-320}{
	\LongitudePlane[{{\u}zplane}]{\angEl}{\u}
	\foreach \r in {0.25,0.5,...,2.25}{
		\draw[{{\u}zplane}, dashed, field line]
		plot (\fieldlinecurve{\r}{\t});
	}
}
% Drawing plane for the B-vectors
\LongitudePlane[bzplane]{\angEl}{0}
\foreach \r in {0.25,0.5,...,2.25}{
	\draw[bzplane, thick, field line]
	plot (\fieldlinecurve{\r}{\t});
}

\begin{scope}[bzplane, very thick, ->, >=stealth]
\draw (\fieldlinecurve{1.25}{-30}) -- +(-30:0.79cm)  node[right] {$\vec{B_{r}}$};
\draw (\fieldlinecurve{1.25}{-30}) -- +(60:0.68cm)   node[right] {$\vec{B_{\theta}}$};
\draw (\fieldlinecurve{1.25}{30})  -- +(-150:0.79cm) node[below] {$\vec{B_{r}}$};
\draw (\fieldlinecurve{1.25}{30})  -- +(120:0.68cm)  node[above] {$\vec{B_{\theta}}$};
\end{scope}
%
\begin{scope}[rotate around={11.1:(0,0)}]
\fill[ball color=white,opacity=0.3] (O) circle (\R); %3D lighting effect
\draw (O) circle (\R);
\DrawLongitudeCircle[\R]{\angAz}      % xzplane
\DrawLongitudeCircle[\R]{\angAz + 90} % vzplane
\DrawLatitudeCircle[\R]{0}            % equator
\DrawLatitudeCircle[\R]{70}           % Latitude 70
\DrawLatitudeCircle[\R]{-70}          % Latitude -70
\end{scope}
%
\coordinate[mark coordinate] (Sm) at (0, \H);
	\coordinate[mark coordinate] (Nm) at (0,-\H);
	\path[xzplane] (Nm) -- +(0,-0.75) coordinate (Nm1) node[below] {$\mathbf{N}_m$}
	(Sm) -- +(0,0.75)  coordinate (Sm1) node[above] {$\mathbf{S}_m$};
	\draw[very thick, dashed]    (Sm) -- (Nm);
	\draw[very thick]            (Sm1) -- (Sm);
	\draw[very thick,->,>=latex'](Nm) -- (Nm1);
	\end{scope}
%	\node[align=justify, text width=14cm, anchor=north west] at (-7,-5.2)
%	{Schematic Earth dipolar magnetic field. The field lines placed in the
%		page plane are drawn as thick lines, those back with dashed lines and
%		the field lines in front of the page with thin lines.};
\end{tikzpicture}
			\vskip-5ex
			\caption{Dipolar Magnetic Field - Cyril Langlois}
		\end{figure}
	\end{minipage}
	\par
	\color{scugreen}此处仅展示示例, 具体操作请查看宏包手册 - ``pgfmanual.pdf''.
\end{frame}

\subsection{动画}
\begin{frame}{Animate}
	编者并不会用此包, 此页面摘自\href{https://zhuanlan.zhihu.com/p/338402487}{\color{scublue}知乎向禹}.\\
	注意, 动画显示需使用Adobe Acrobat等支持JavaScript的PDF浏览器查看(我们学校的电脑上应该有).\\
	这里放一个大佬做的例子(弹簧振子):
	\begin{center}
		\begin{animateinline}[loop]{20}%
			\multiframe{160  }{rx=0+5}{
				\begin{tikzpicture}[scale=1.5]
				\pgfmathsetmacro{\x}{4+1*sin(\rx)}	
				\coordinate  (P) at (\x,0);	
				\draw [decorate,decoration={coil,segment length= \x pt, pre length=2mm, post length=2mm
					,amplitude=2mm}] (0,0)--(P);
				\shade[ball color=black](P)circle (2mm);
				\fill [pattern = north east lines] (0,0.-0.2) -- ++ (0,0.4) -- ++ (-0.1,0) -- ++(0,-0.4) --cycle
				(0,-0.2) -- ++(0,-0.1) -- ++ (5.5,0) -- ++ (0,0.1) -- cycle;
				\draw (0,0.2) -- (0,-0.2) -- (5.5,-0.2);
				\end{tikzpicture} 	    }
		\end{animateinline}
	\end{center}
	放过我吧, 关于Animate插入GIF动图, 明确告诉你, 不能, 所以具体方法请百度"LaTeX animate", 蟹蟹理解.\par
	\color{scugreen}此处仅展示示例, 具体操作请查看宏包手册 - ``animate.pdf''.
\end{frame}

% --------节: FAQ--------
% ----------------
\section{FAQ}
\subsection{编译相关}
\begin{frame}{Info.}
  \textbf{本小节将介绍编译相关的常见问题.}
  \begin{multicols}{2}
    \begin{itemize}
      \item Establish: \textcolor{scugreen}{v1.3b (2022/04/13)}
    \end{itemize}
  \end{multicols}
  \mycopyright
\end{frame}

\begin{frame}[allowframebreaks]
  \structure{Headline, Footline及页码未显示或未正常显示}\par
  原因: 初次进行编译或出现新增帧. | \textbf{请再次进行编译操作(为保障编译速度, 非最终版本请无视此问题).}\par\vspace{2ex}
  \structure{参考文献列表空白}\par
  可能原因: 编辑器未正常启动biber或bibtex引擎. | \textbf{请手动运行相关命令编译bib数据库.}\par\vspace{2ex}
\end{frame}

% --------节: 参考文献--------
% ----------------
\section{参考文献}
\subsection{参考文献}
\begin{frame}[allowframebreaks]{文献目录}
	\nocite{*}
	\printbibliography[heading=none]
\end{frame}

% --------节: 致谢--------
% ----------------
\section{致谢}
\subsection{致谢}
\begin{frame}{致谢}
	\begin{center}
		本模板参考了Beamer, Tcolorbox等手册, 感谢宏包原作者及维护者\\[1ex]
		本模板参考了知乎, Stack Overflow等平台回答, 感谢相关问题解答者\\[1ex]
		本模板使用了开源字体——楷体: 霞鹜文楷(Github \href{https://github.com/lxgw/LxgwWenKai/}{\color{scublue}LxgwWenKai} 项目), 黑体: Source Han Sans (Github \href{https://github.com/adobe-fonts/source-han-sans/}{\color{scublue}source-han-sans} 项目), 感谢字体设计师设计的优秀字体\\[1ex]
		本模板参考了中国科学技术大学Beamer模板(Github \href{https://github.com/ysx2000/USTCBeamerSX/}{\color{scublue}USTCBeamerSX} 项目), 感谢原作者提供部分设计思路\\[1ex]
		本模板参考了清华大学Beamer模板(Github \href{https://github.com/tuna/THU-Beamer-Theme/}{\color{scublue}THU-Beamer-Theme} 项目), 中国科学技术大学Beamer模板(Github \href{https://github.com/ustctug/ustcbeamer/}{\color{scublue}ustcbeamer} 项目), 感谢原作者设计的优秀模板\\[1ex]
		
		若在使用过程中发现些许Bug, 感谢诸位理解, 在此也希望诸位能先行尝试多次编译\\[1ex]
		万分感谢诸位批评指正, 感谢诸位对模板及对制作者的支持!
	\end{center}
\end{frame}

\begin{frame}
	\vskip6ex
	\begin{columns}
		\begin{column}[c]{0.4\columnwidth}
			\centering\large
			用户手册到此结束.\\[2ex]
			\normalsize This is the end of the User's Manual.
		\end{column}
		\begin{column}[c]{0.6\columnwidth}
			\centering
			\Large 感谢浏览本手册!\\[2ex]
			Happy LaTeXing!
		\end{column}
	\end{columns}
	\centering\vskip4ex
	SCU Beamer Slide-demo, Rev 1.3e
\end{frame}

\appendix
% --------节: 附录A scu 宏包--------
% ----------------
% ----------------
% 节: 附录A scu 宏包
% ----------------
\section{附录A}
\subsection{scu 主题选项预览}
\begin{frame}{Info.}
	\textbf{本小节将展示部分 scu 主题选项的预览效果.}
	\begin{multicols}{2}
		\begin{itemize}
			\item \myestablish{v1.3e}{2024/10/31}
			\item \myupdate{v1.3e}{2024/10/31}
		\end{itemize}
	\end{multicols}
	注: 本小节仅对部分主题参数进行了演示, 您可点击标题栏 \texttt{BACK} 字样返回对应的宏包参数说明处.\par
	\mycopyright
\end{frame}

\begin{frame}{颜色主题预览 \scugoback{back:ColorDisplay}{BACK}}\label{goto:ColorDisplay}
  \structure{\cmd{usetheme}\oarg*{ColorDisplay=\Arg{value}}\marg*{scu}}
	\begin{columns}[T, onlytextwidth]
    \begin{column}{.5\textwidth}
      \structure{锦锈红} \Arg{value}\texttt{=JXred}
      \begin{figure}[h]
        \centering
        \includegraphics[width=\columnwidth]{manual-sec/manual-demo/appendix-themescu-ColorDisplay-1.pdf}
      \end{figure}
    \end{column}
    \begin{column}{.5\textwidth}
      \structure{宝石蓝} \Arg{value}\texttt{=BSblue}
      \begin{figure}[h]
        \centering
        \includegraphics[width=\columnwidth]{manual-sec/manual-demo/appendix-themescu-ColorDisplay-2.pdf}
      \end{figure}
    \end{column}
  \end{columns}
\end{frame}

\begin{frame}{小节迷你帧预览 \scugoback{back:Miniframes}{BACK}}\label{goto:Miniframes}
  \vspace*{-6ex}
  \begin{figure}[h]
    \begin{tikzpicture}
      \node[anchor=south west] (img) at (0,10pt) {\includegraphics[height=.88\textheight]{manual-sec/manual-demo/appendix-themescu-Miniframes.pdf}};
      \node[anchor=west] at (8, 211.4pt) {\structure{\cmd{usetheme}\oarg*{Miniframes=\Arg{value}}\marg*{scu}}};
      % --------- First ----------
      \def\xFir{28.8pt}
      \def\yFir{211.4pt}
      \draw[scublue, thin] (\xFir-6pt, \yFir-2pt) rectangle (\xFir+6pt, \yFir+2pt); % 以 (x, y) 为中心的框
      \draw[scublue, thin] (\xFir+6pt, \yFir-2pt) -- (\xFir+28pt, \yFir-13pt); % 斜线
      \draw[scublue, thin] (\xFir+28pt, \yFir-13pt) -- (8, \yFir-13pt); % 水平线
      \node[anchor=west] (NodeOne) at (8, \yFir-13pt) {\structure{迷你帧跟随} \Arg{value}\texttt{=follow}};
      \node[anchor=west] at ([yshift=-12pt]NodeOne.west) {迷你帧跟随当前小节标题(如此处为“小节迷你帧”)};
      % ---------- Second ----------
      \def\xSec{149.5pt}
      \def\ySec{154.2pt}
      \draw[scublue, thin] (\xSec-6pt, \ySec-2pt) rectangle (\xSec+6pt, \ySec+2pt); % 以 (x, y) 为中心的框
      \draw[scublue, thin] (\xSec+6pt, \ySec) -- (8, \ySec); % 水平线
      \node[anchor=west] (NodeTwo) at (8, \ySec) {\structure{迷你帧分离} \Arg{value}\texttt{=separate}};
      \node[anchor=west] at ([yshift=-12pt]NodeTwo.west) {迷你帧与小节标题分离(位于节导航栏下方且右侧对齐)};
      % ---------- Third ----------
      \def\xThiO{28.8pt}
      \def\yThiO{105.5pt}
      \draw[scublue, thin] (\xThiO-6pt, \yThiO-2pt) rectangle (\xThiO+6pt, \yThiO+2pt); % 以 (x, y) 为中心的框
      \draw[scublue, thin] (\xThiO+6pt, \yThiO-2pt) -- (\xThiO+28pt, \yThiO-13pt); % 斜线
      \draw[scublue, thin] (\xThiO+28pt, \yThiO-13pt) -- (8, \yThiO-13pt); % 水平线
      \def\xThiT{149.5pt}
      \def\yThiT{101.3pt}
      \draw[scublue, thin] (\xThiT-6pt, \yThiT-2pt) rectangle (\xThiT+6pt, \yThiT+2pt); % 以 (x, y) 为中心的框
      \draw[scublue, thin] (\xThiT+6pt, \yThiT-2pt) -- (\xThiT+19.6pt, \yThiT-8.8pt); % 斜线
      \node[anchor=west] (NodeThree) at (8, \yThiO-13pt) {\structure{取消迷你帧} \Arg{value}\texttt{=negate}};
      \node[anchor=west] at ([yshift=-12pt]NodeThree.west) {页眉上无迷你帧显示};
      % ---------- Forth ----------
      \def\xForO{146.6pt}
      \def\yForO{48.2pt}
      \draw[scublue, thin] (\xForO-8pt, \yForO-2pt) rectangle (\xForO+8pt, \yForO+2pt); % 以 (x, y) 为中心的框
      \draw[scublue, thin] (\xForO+8pt, \yForO-2pt) -- (\xForO+20pt, \yForO-8pt); % 斜线
      \def\xForT{49pt}
      \def\yForT{31.4pt}
      \draw[scublue, thin] (\xForT-6pt, \yForT-2.8pt) rectangle (\xForT+6pt, \yForT+2.8pt); % 以 (x, y) 为中心的框
      \draw[scublue, thin] (\xForT+6pt, \yForT+2.8pt) -- (\xForT+18pt, \yForT+8.8pt); % 斜线
      \draw[scublue, thin] (\xForT+18pt, \yForT+8.8pt) -- (8, \yForT+8.8pt); % 水平线
      \node[anchor=west, align=left, yshift=-4pt] at (8, \yForT+8.8pt) {自v1.3e起,迷你帧支持点击跳转\\若小节中页面数量过多且设置迷你帧跟随小节标题时\\\alert{迷你帧可自适应换行}\\反之设置迷你帧与小节标题分离则单行显示};
    \end{tikzpicture}
  \end{figure}
  \vspace*{-6ex}
\end{frame}

\begin{frame}{导航工具栏详情 \scugoback{back:NavigationTool}{BACK}}\label{goto:NavigationTool}
  \structure{\cmd{usetheme}\oarg*{NavigationTool=\Arg{value}}\marg*{scu}}\vspace*{-2.5ex}
  \begin{figure}[h]
    \centering
    \includegraphics[width=\textwidth]{manual-sec/manual-demo/appendix-themescu-NavigationTool.pdf}
  \end{figure}

  \structure{小节及小节跳转工具} 标记为1 - \alert{\hskip.4pt\setfontscu{5}\faCaretLeft\hskip1.6pt\faStream\hskip1.6pt\faCaretRight}\vspace*{1ex}

  \begin{tabular}{l>{\raggedright\arraybackslash}p{.02\textwidth}>{\raggedright\arraybackslash}p{.35\textwidth}l}
    核心工具 & {\setfontscu{5}\faStream} & 前往当前小节开始 & -\\
    附加工具 & {\setfontscu{5}\faCaretLeft} & 前往上一小节末尾 & -\\
    附加工具 & {\setfontscu{5}\faCaretRight} & 前往下一小节开始 & -\\
  \end{tabular}\vspace*{1ex}

  \structure{查询及首尾跳转工具} 标记为2 - \alert{\hskip.4pt\setfontscu{5}\faCaretLeft\hskip-.05em\faCaretLeft\hskip1.6pt\setfontscu{4.5}\faSearch\hskip1.6pt\setfontscu{5}\faCaretRight\hskip-.05em\faCaretRight}\vspace*{1ex}

  \begin{tabular}{l>{\raggedright\arraybackslash}p{.02\textwidth}>{\raggedright\arraybackslash}p{0.35\textwidth}l}
    核心工具 & {\setfontscu{4.5}\faSearch} & 查找(需要使用Adobe Acrobat) & Acrobat快捷键: Ctrl + F\\
    附加工具 & {\setfontscu{5}\faCaretLeft\hskip-.05em\faCaretLeft} & 前往初始页(不包含附录) & -\\
    附加工具 & {\setfontscu{5}\faCaretRight\hskip-.05em\faCaretRight} & 前往结尾页(不包含附录) & -\\
  \end{tabular}\vspace*{1ex}

  \structure{放映及历史跳转工具} 标记为3 - \alert{\hskip.4pt\setfontscu{4.5}\faReply\hskip1.6pt\setfontscu{5}\faExpand\hskip1.6pt\setfontscu{4.5}\reflectbox{\faReply}}\vspace*{1ex}

  \begin{tabular}{l>{\raggedright\arraybackslash}p{.02\textwidth}>{\raggedright\arraybackslash}p{0.35\textwidth}l}
    核心工具 & {\setfontscu{5}\faExpand} & 全屏或退出全屏(需要使用Adobe Acrobat) & Acrobat快捷键: Ctrl + L\\
    附加工具 & {\setfontscu{4.5}\faReply} & 前往上一视图(需要使用Adobe Acrobat) & Acrobat快捷键: Alt + 向左箭头\\
    附加工具 & {\setfontscu{4.5}\reflectbox{\faReply}} & 前往下一视图(需要使用Adobe Acrobat) & Acrobat快捷键: Alt + 向右箭头\\
  \end{tabular}
\end{frame}

\subsection{scu 主题项目结构}
\begin{frame}{Info.}
	\textbf{本小节将展示 scu 主题的项目结构.}
	\begin{multicols}{2}
		\begin{itemize}
			\item \myestablish{v1.3e}{2024/10/31}
			\item \myupdate{v1.3e}{2024/10/31}
		\end{itemize}
	\end{multicols}
	注: 自 \textcolor{scugreen}{v1.3e (2024/10/31)} 起, 本小节已替代手册中\alert{\textbf{使用注意}}小节的\alert{项目结构}板块.\par
	\mycopyright
\end{frame}
\begin{frame}{main 分支文件结构 \scugoback{back:BranchMain}{BACK}}\label{goto:BranchMain}
  从 release 中下载的版本也具有如下的文件结构.
  \begin{multicols}{2}
    \setlength{\DTbaselineskip}{10pt}
    \dirtree{%
    .1 \alert{SCU Beamer Theme}/.
    .2 fonts/\DTcomment{\alert{字体存放}[资源]}.
    .2 image/\DTcomment{\alert{图像存放}[资源]}.
    .2 mintedbuild/\DTcomment{\alert{minted 缓存}[缓存]}.
    .2 resources/\DTcomment{\alert{\textbf{scu 主题素材}}[主题]}.
    .3 SCUbuilding.png\DTcomment{\text{建筑}[主题]}.
    .3 SCUlogo\_name.pdf\DTcomment{\text{logo + 校名}[主题]}.
    .3 SCUname.pdf\DTcomment{\text{校名}[主题]}.
    .3 SCUverify.png\DTcomment{\text{校训}[主题]}.
    .3 background.png.
    .3 backgroundofsubsectiontocpage.png.
    .3 backgroundoftitlepage(Empty).png.
    .3 backgroundoftitlepage(Light).png.
    .3 backgroundoftitlepage.png.
    .2 sourcecode/\DTcomment{\alert{代码存放}[资源]}.
    }
    \dirtree{%
    .1 \alert{SCU Beamer Theme}/.
    .2 beamercolorthemescu.sty\DTcomment{\alert{\textbf{颜色主题}}[主题]}.
    .2 beamerinnnerthemescu.sty\DTcomment{\alert{\textbf{内部主题}}[主题]}.
    .2 beamerouterthemescu.sty\DTcomment{\alert{\textbf{外部主题}}[主题]}.
    .2 beamerthemescu.sty\DTcomment{\alert{\textbf{scu核心主题}}[主题]}.
    .2 .gitignore.
    .2 LICENSE.
    .2 README.md.
    .2 main.tex\DTcomment{\alert{cn模式tex文件}[示例]}.
    .2 main.pdf\DTcomment{\alert{cn模式pdf文件}[示例]}.
    .2 main-en.tex\DTcomment{\alert{en模式tex文件}[示例]}.
    .2 main-en.pdf\DTcomment{\alert{en模式pdf文件}[示例]}.
    .2 manual.pdf\DTcomment{\alert{\textbf{用户手册}}[手册]}.
    .2 ref.bib\DTcomment{\alert{\textbf{文献库}}[资源]}.
    }
  \end{multicols}
\end{frame}

\begin{frame}{manual 分支文件结构 \scugoback{back:BranchManual}{BACK}}\label{goto:BranchManual}
  如下的示例演示资源中,pdf文件并不完全是tex直接编译所得,而是编译后经过inkscape拼接、裁剪和压缩生成.
  \begin{multicols}{2}
    \setlength{\DTbaselineskip}{10pt}
    \dirtree{%
    .1 \alert{SCU Beamer Theme Manual}/.
    .2 manual-sec/\DTcomment{\alert{手册分节tex文件}[章节]}.
    .3 manual-demo/\DTcomment{\alert{手册中示例演示资源}[演示]}.
    .4 \text{...}.tex\DTcomment{\alert{示例tex文件}[演示]}.
    .4 \text{...}.pdf\DTcomment{\alert{示例pdf文件}[演示]}.
    .3 base-settings.tex\DTcomment{\alert{基础设置}[章节]}.
    .3 appendix-themescu.tex\DTcomment{\alert{附录A}[章节]}.
    }\columnbreak
    \dirtree{%
    .1 \alert{SCU Beamer Theme Manual}/.
    .2 manual.tex\DTcomment{\alert{用户手册tex文件}[手册]}.
    .2 manual.pdf\DTcomment{\alert{用户手册pdf文件}[手册]}.
    }
  \end{multicols}
\end{frame}

\section{附录B}
\begin{frame}{beamer宏包}\label{frame:abc}
	测试
\end{frame}

\begin{frame}{beamer宏包包}\label{frame:adc}
	测试
\end{frame}

\end{document}

%% End of file `manual.tex'.